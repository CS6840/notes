\Lecture{Prasun Kumar}{Jan 12, 2012}{5}{$\FP$ vs $\#\P$ question}
%\theme{ Between $\P$ and $\PSPACE$.}
%\lectureplan{ $\FP$ vs $\#\P$ question, a counter part in the decision world. The class PP. PP vs P is equivalent to FP vs $\#\P$.}

We know $(\#\P = \FP) \implies (\P = \NP)$. In the last lecture we left the question about the converse.
i.e., is it true that, $(\P = \NP) \implies (\#\P = \FP)$?

In this lecture, we will show a weaker version of the above containment.
We will define a new class of languages $\PP$ (probabilistic polynomial time)
in decision world, that contains $\NP$, and will show that if we make a stronger assumption $\PP = \P$ (seemingly stronger than $\NP = \P$) we get the required implication.  We will show in particular that $\PP = \P \iff \#\P = \FP$.

\section{Complexity Class \PP}
For exploring the complexity class, we revisit the definition of $\NP$.
With respect to a "branching" Turing machine(NTM) (which can branch at each computation by using a guess bit), we have two different classes of languages as follows, which differs in their acceptance conditions. We denote for a machine $M$, by $\#acc_M(x)$ as the number of accepting paths. Let the machine run for $p(n)$ time on each path.
\begin{itemize}
\item $\NP$ = $\{$L : $\exists$ branching TM $M_1$, x $\in$ L $\iff \#acc_{M_1}(x) \ge 1 \}$.
\item $\coNP$ = $\{$L : $\exists$ branching TM $M_2$, x $\in$ L $\iff \#acc_{M_2}(x) \ge 2^{p(n)} \}$.
\end{itemize}

In terms of number of accepting paths on input $x$, the above classes
represent the extremes.  On one side, in $\NP$ we talk of atleast one
accepting path, and on the other side, in $\coNP$ we talk of all
accepting paths. To understand the structure of these classes we can
ask the variant of this as : what different class of languages do we
get if we set the accepting condition as the number of accepting paths
being more than a fraction of the total number of paths on input {\em
  x}. A simpler situation is to consider the fraction to be half and the
corresponding class of languages is called \PP (probabilistic
polynomial) which is defined as follows. We will come across other variants in
the later lectures.

\begin{definition}
$\PP$ =$\{$L : $\exists$ NTM $M$, x $\in$ L $\iff \#acc_{M_1}(x) > 2^{p(n)-1} \}$
\end{definition}

We first understand this complexity class with respect to the other ones we have already seen. We start with the following proposition.

\begin{proposition}
\label{prop}
$\NP \subseteq \PP$.
\end{proposition}
\begin{proof}
Let $L \in \NP$  via a nondeterministic turing machine $M$,  $ x \in L \iff M$ has atleast one accepting path on $x$. 
Our aim is to give another turing machine $M'$ for the same language $L$ such that
$x \in L \iff M'$ has more than half of the total number of paths as accepting paths on $x$.\\
Description of $M'$ (on input $x$):
\begin{enumerate}
\item Simulate $M$ on $x$.
\item If $M$ accepts then choose one bit nondeterministically and accept in both branches.
\item If $M$ rejects then choose one bit nondeterministically,accept in one branch and reject in other.
\end{enumerate}

As we remarked in the last lecture, we can assume without loss of generality that the height of computation tree of $M$ on $x$ is exactly $p(n)$ for some polynomial $p$. Since the total number of paths for $M'$ on input $x$ is $2^{p(n)+1}$, it suffices to prove that $x \in L \iff \#acc_M(x) > 2^{p(n)}$.
Let $x \in L$. Since $M$ has at least one accepting path on $x$. Since $M'$, by construction, creates an imbalance (in count) between the number of accepting and rejecting paths precisely when $M$ accepts,  the number of accepting paths in $M'$ will be more than the number of rejecting paths. Thus $\#acc_M(x) > 2^{p(n)}$. \\
If  $x \not\in L$, all paths reject in $M$ on input $x$, and thus  number of accepting and rejecting
paths in $M'$ on input $x$ are exactly equal and each of themsame is equal to $2^{p(n)}$.
\end{proof}

A similar proof will also show that $\coNP$ is contained in
$\PP$. Thus we have the following containment relationship among
different classes of languages
 
\begin{center}
\PSPACE \\ 
$\vert$ \\
\PP \\
$\vert$ \\
\NP \\
$\vert$ \\
\P
\end{center}

Now we will prove the main theorem of the lecture which characterizes the $\FP$ vs $\#\P$ question in the decision world.

\begin{theorem}
 $(\#\P = \FP) \iff (\PP = \P)$
\end{theorem}
\begin{proof}
($\Rightarrow$) Assume $\#\P = \FP$, our aim is to prove $\PP = \P$. The reverse containment follows since we know $\NP \subseteq \PP$ by proposition\ref{prop}. Now we show that $\PP \subseteq \P$ . Let $L \in \PP$
via a machine $M$ running in time $p(n)$ for some polynomial $p$, such that $x \in L \iff \#acc_M(x) > 2^{p(n)-1}$.
Define the function, $f(x) = \#acc_M(x)$. By definition $f \in \#\P$ and hence $f \in \FP$. There is a deterministic polynomial time Turing machine $N$ which on input $x$ outputs the value of $f(x)$. Given an $x$, to test whether it is in $L$ or not, it suffices to test whether whether the MSB of the binary representation of $f(x)$ is 1 or not. This can be done by simply running the machine $N$ on $x$ and testing the MSB of the output. Hence $L \in P$.

($\Leftarrow$) Assume $\PP = \P$, our aim is to prove $\#\P = \FP$. The reverse direction is easy. $\FP \subseteq \#\P$ as we argued in the last lecture. To show the forward direction, let $f\in \#\P$ via $M$ such that $\forall x \in \Sigma^*$, $f(x)=\# accept_M(x)$. Note that the naive approach to compute $f(x)$ is to compute the number of accepting paths in computation tree of height $p(n)$ will take exponential time. But it suffices to find the minimum $0 \le k \le 2^{p(n)}$ such that :
\begin{equation}
\label{eqn}
k+\#acc_M(x) > 2^{p(n)}
\end{equation}
Indeed, the minimum is achieved when $k = k_{min} = 2^{p(n)} - \#acc_M(x) + 1$. Thus $\#acc_M(x) = 2^{p(n)} - k_{min} +1$. This is an indirect way of finding out $\#acc_M(x)$ and we are moving towards using our assumption that $\PP = \P$.

We have  search problem in hand; to search for the minimum $k$($k_{min}$) satisfying equation (\ref{eqn}). We do this by binary search over the range $0 \le k \le 2^{p(n)}$. We solve the decision problem first. {\sf Given $x$ and $k$, check if $k+\#acc_M(x) > 2^{p(n)}$}.

For this, we construct an another Turing machine $N$ such that the number of accepting paths is exactly $k+\#acc_M(x)$ and the total number of paths is $2^{p(n)+1}$. Assume such a construction exists. Define a language $A \subset \Sigma^*$ such that $x \in A \iff k+\#acc_M(x) > 2^{p(n)}$. Thus $x \in A \iff \#acc_N(x) > 2^{p(n)}$. This implies that $A \in PP$ (via the machine $N$ !) and hence $A \in P$ by assumption. Given $x \in \Sigma^*$ we can test if $k+\#acc_M(x) > 2^{p(n)}$ by testing if $x \in A$ or not, which can be done in polynomial time. Now we can do this construction and simulation through a binary search in order to find the minimum value of $k$ that satisfies our inequality.

To complete the proof, we give the construction of $N$ (for a given $k$). 
We first construct a machine $M_k$ that runs in time $p(n)$ and has exactly $k$ accepting paths. 
We slightly modify the idea in the previous lecture to do this. Let $\ell = \lceil \log k \rceil$.
The machine $M_k$ guesses a $y \in \{0,1\}^{p(n)}$ and {\em accepts} if the first $\ell$ bits of $y$ in binary represents a number less than $k$ and the last $p(n)-\ell$  bits is all-zero (lexicographically first path) and {\em reject} otherwise.

We combine $M_k$ and $N$ (both using exactly $p(n)$ non-deterministic bits), to get the machine $N$. The machine $N$ on input $x$ guesses $1$ bit and on the $0$-branch it simulates $M_k$ on $x$ and on the $1$-branch it simulates $M$ on $x$. Clearly $\#acc_N(x) = \#acc_{M_k}(x) + \#acc_{M}(x) = k+\#acc_{M}(x)$. The length of each path is exactly $p(n)+1$ and hence total number of paths is $2^{p(n)+1}$.
\end{proof}



