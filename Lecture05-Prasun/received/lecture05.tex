\documentclass[11pt]{article}
\usepackage{amsmath,amssymb,amsfonts,amsthm,complexity}

\newcommand{\CYCLE}{{\sf CYCLE~}}
\newcommand{\FPSPACE}{{\sf FPSPACE~}}
\usepackage{tikz}
\usetikzlibrary{calc,through,backgrounds,decorations.pathmorphing}

%%%%%%%%%%%%%%%%%%%%%%%%%%%%%%%%%%%%%%%%%%%%%%%%%%%%%%
%
% This file should be called  preamble.tex 
% Your main LaTeX file should look like :
%
%	\documentclass[11pt]{article}
%	\usepackage{amsmath,amssymb,amsfonts,amsthm,complexity}
%
%	%%%%%%%%%%%%%%%%%%%%%%%%%%%%%%%%%%%%%%%%%%%%%%%%%%%%%%
%
% This file should be called  preamble.tex 
% Your main LaTeX file should look like :
%
%	\documentclass[11pt]{article}
%	\usepackage{amsmath,amssymb,amsfonts,amsthm,complexity}
%
%	%%%%%%%%%%%%%%%%%%%%%%%%%%%%%%%%%%%%%%%%%%%%%%%%%%%%%%
%
% This file should be called  preamble.tex 
% Your main LaTeX file should look like :
%
%	\documentclass[11pt]{article}
%	\usepackage{amsmath,amssymb,amsfonts,amsthm,complexity}
%
%	\input{preamble.tex}
%	\begin{document}
%
%	\lecture{LECTURE NO.}{TITLE OF SCRIBE}{DATE}{Jayalal Sarma M.N.}{YOUR NAME}
%	\theme{THEME FOCUS}
%	\lectureplan{A BRIEF DESCRIPTION OF THE LECTURE}
%	
%	\section{TOPIC 1}
%	...
%	\section{TOPIC 2}
% 	...
%	\end{document}
% If there is not title, leave it as {}
 

\newtheorem{theorem}{Theorem}
\newtheorem{corollary}[theorem]{Corollary}
\newtheorem{lemma}[theorem]{Lemma}
\newtheorem{observation}[theorem]{Observation}
\newtheorem{proposition}[theorem]{Proposition}
\newtheorem{claim}[theorem]{Claim}
\newtheorem{fact}[theorem]{Fact}
\newtheorem{example}[theorem]{Example}
\newtheorem{assumption}[theorem]{Assumption}

\theoremstyle{definition}
\newtheorem{definition}[theorem]{Definition}

\theoremstyle{remark}
\newtheorem{remark}[theorem]{Remark}

% Setting theorem style back for theorems defined in main file.
\theoremstyle{plain}

%\newenvironment{proof}{\noindent{\bf Proof}\hspace*{1em}}{\qed\bigskip}
\newenvironment{proof-sketch}{\noindent{\bf Sketch of Proof}\hspace*{1em}}{\qed\bigskip}
\newenvironment{proof-idea}{\noindent{\bf Proof Idea}\hspace*{1em}}{\qed\bigskip}
\newenvironment{proof-of-lemma}[1]{\noindent{\bf Proof of Lemma #1}\hspace*{1em}}{\qed\bigskip}
\newenvironment{proof-attempt}{\noindent{\bf Proof Attempt}\hspace*{1em}}{\qed\bigskip}
\newenvironment{proofof}[1]{\noindent{\bf Proof}
of #1:\hspace*{1em}}{\qed\bigskip}

%%%%%%%%%%%%%%%%%%%%%%%%%%%%%%%%%%%%%%%%%%%%%%%%%%%
% Useful Complexity Classes
%%%%%%%%%%%%%%%%%%%%%%%%%%%%%%%%%%%%%%%%%%%%%%%%%%%
\newcommand{\ntime}{\hbox{NTIME}}
\newcommand{\nspace}{\hbox{NSPACE}}
\newcommand{\conspace}{\hbox{co-NSPACE}}
\newcommand{\np}{\hbox{NP}}
\newcommand{\pspace}{\hbox{PSPACE}}
\newcommand{\lspace}{\hbox{L}}
\newcommand{\conp}{\hbox{coNP}}
\newcommand{\exptime}{\hbox{EXPTIME}}
\newcommand{\elem}{\hbox{E}}
\newcommand{\nl}{\hbox{NL}}
\newcommand{\bpp}{\hbox{BPP}}
\newcommand{\nregexp}{\hbox{NREGEXP}}
\newcommand{\tqbf}{\hbox{TQBF}}
\newcommand{\threesat}{\hbox{3SAT}}
\newcommand{\cvp}{\hbox{CVP}}
\newcommand{\stconn}{\hbox{STCONN}}
\newcommand{\ispath}{\hbox{ISPATH}}

%\newcommand{\class}{\hbox{$\mathbb{C}$}} 
%\newcommand{\class}{\hbox{$\mathbf{C}$}} 

\newcommand{\lep}{\leq _{\hbox{P}}}
\newcommand{\lel}{\leq _{\hbox{L}}}
\newcommand{\aspace}[1]{{\rm ASPACE}(#1)}
\newcommand{\atime}[1]{{\rm ATIME}(#1)}
\newcommand{\dtime}[1]{{\rm DTIME}(#1)}
\newcommand{\spa}[1]{{\rm SPACE}(#1)}
\newcommand{\ti}[1]{{\rm TIME}(#1)}
\newcommand{\ap}{{\rm AP}}
\newcommand{\al}{{\rm AL}}


%%%%%%%%%%%%%%%%%%%%%%%%%%%%%%%%%%%%%%%%%%%%%%%%%%%
% Useful Macros
%%%%%%%%%%%%%%%%%%%%%%%%%%%%%%%%%%%%%%%%%%%%%%%%%%%
\renewcommand{\Pr}[1]{\mathify{\mbox{Pr}\left[#1\right]}}
\newcommand{\Exp}[1]{\mathify{\mbox{Exp}\left[#1\right]}}
\newcommand{\bigO}O
\newcommand{\set}[1]{\mathify{\left\{ #1 \right\}}}
\def\half{\frac{1}{2}}

\def\implies{\Rightarrow}
\def\prob#1#2{{\mathop{{\rm Prob}}_{#1}}\left[#2 \right]}
\def\var#1#2{{\mathop{{\rm Var}}_{#1}}[#2]}
\def\expec#1#2{{\mathop{{\rm E}}_{#1}}[#2]}
\def\sizeof#1{\left| #1\right|}
\def\setof#1{\left\{ #1\right\}  }

\newcommand{\F}{{\mathbb{F}}}
\newcommand{\Z}{{\mathbb{Z}}}
%\newcommand{\qed}{\rule{7pt}{7pt}}

% \makeatletter
% \@addtoreset{figure}{section}
% \@addtoreset{table}{section}
% \@addtoreset{equation}{section}
% \makeatother

\newcommand{\FOR}{{\bf for}}
\newcommand{\TO}{{\bf to}}
\newcommand{\DO}{{\bf do}}
\newcommand{\WHILE}{{\bf while}}
\newcommand{\AND}{{\bf and}}
\newcommand{\IF}{{\bf if}}
\newcommand{\THEN}{{\bf then}}
\newcommand{\ELSE}{{\bf else}}
\newcommand{\N}{\mathbb{N}}

% \renewcommand{\thefigure}{\thesection.\arabic{figure}}
% \renewcommand{\thetable}{\thesection.\arabic{table}}
% \renewcommand{\theequation}{\thesection.\arabic{equation}}

% Calligraphic letters
\newcommand{\calA}{{\cal A}}
\newcommand{\calB}{{\cal B}}
\newcommand{\calC}{{\cal C}}
\newcommand{\calD}{{\cal D}}
\newcommand{\calE}{{\cal E}}
\newcommand{\calF}{{\cal F}}
\newcommand{\calG}{{\cal G}}
\newcommand{\calH}{{\cal H}}
\newcommand{\calI}{{\cal I}}
\newcommand{\calJ}{{\cal J}}
\newcommand{\calK}{{\cal K}}
\newcommand{\calL}{{\cal L}}
\newcommand{\calM}{{\cal M}}
\newcommand{\calN}{{\cal N}}
\newcommand{\calO}{{\cal O}}
\newcommand{\calP}{{\cal P}}
\newcommand{\calQ}{{\cal Q}}
\newcommand{\calR}{{\cal R}}
\newcommand{\calS}{{\cal S}}
\newcommand{\calT}{{\cal T}}
\newcommand{\calU}{{\cal U}}
\newcommand{\calV}{{\cal V}}
\newcommand{\calW}{{\cal W}}
\newcommand{\calX}{{\cal X}}
\newcommand{\calY}{{\cal Y}}
\newcommand{\calZ}{{\cal Z}}


% Some macro's from Speilman's course.

\makeatletter
\def\fnum@figure{{\bf Figure \thefigure}}
\def\fnum@table{{\bf Table \thetable}}
\long\def\@mycaption#1[#2]#3{\addcontentsline{\csname
  ext@#1\endcsname}{#1}{\protect\numberline{\csname 
  the#1\endcsname}{\ignorespaces #2}}\par
  \begingroup
    \@parboxrestore
    \small
    \@makecaption{\csname fnum@#1\endcsname}{\ignorespaces #3}\par
  \endgroup}
\def\mycaption{\refstepcounter\@captype \@dblarg{\@mycaption\@captype}}
\makeatother

\newcommand{\figcaption}[1]{\mycaption[]{#1}}
\newcommand{\tabcaption}[1]{\mycaption[]{#1}}


%%%%%%%%%%%%%%%%%%%%%%%%%%%%%%%%%%%%%%%%%%%%%%%%%%%%%%%%%%%%%%%%%%%%%%
% Feel free to ignore the rest of this file.

%%%%%%%%%%%%%%%%%%%%%%%%%%%%%%%
% Margins and page indentations
% DO NOT EDIT
%%%%%%%%%%%%%%%%%%%%%%%%%%%%%%
\textwidth=6in
\oddsidemargin=0.25in
\evensidemargin=0.25in
\topmargin=-0.1in
\footskip=0.8in
\parindent=0.0cm
\parskip=0.3cm
\textheight=8.00in
\setcounter{tocdepth} {3}
\setcounter{secnumdepth} {2}
\sloppy


\newcommand{\handout}[5]{
   \renewcommand{\thepage}{#1-\arabic{page}}
   \noindent
   \begin{center}
   \framebox{
      \vbox{
    \hbox to 5.78in { {\bf CS6840: Advanced Complexity Theory} \hfill #2 }
       \vspace{4mm}
       \hbox to 5.78in { {\Large \hfill #5  \hfill} }
       \vspace{2mm}
       \hbox to 5.78in { {\em #3 \hfill #4} }
      }
   }
   \end{center}
   \vspace*{4mm}
}

\newcommand{\lecture}[5]{\handout{#1}{#3}{Lecturer:~#4}{Scribe: #5}{Lecture~#1~: #2}}

\newcommand{\lectureplan}[1]{{\sc Lecture Plan:}#1\\}
\newcommand{\theme}[1]{{\sc Theme:} #1\\}

%	\begin{document}
%
%	\lecture{LECTURE NO.}{TITLE OF SCRIBE}{DATE}{Jayalal Sarma M.N.}{YOUR NAME}
%	\theme{THEME FOCUS}
%	\lectureplan{A BRIEF DESCRIPTION OF THE LECTURE}
%	
%	\section{TOPIC 1}
%	...
%	\section{TOPIC 2}
% 	...
%	\end{document}
% If there is not title, leave it as {}
 

\newtheorem{theorem}{Theorem}
\newtheorem{corollary}[theorem]{Corollary}
\newtheorem{lemma}[theorem]{Lemma}
\newtheorem{observation}[theorem]{Observation}
\newtheorem{proposition}[theorem]{Proposition}
\newtheorem{claim}[theorem]{Claim}
\newtheorem{fact}[theorem]{Fact}
\newtheorem{example}[theorem]{Example}
\newtheorem{assumption}[theorem]{Assumption}

\theoremstyle{definition}
\newtheorem{definition}[theorem]{Definition}

\theoremstyle{remark}
\newtheorem{remark}[theorem]{Remark}

% Setting theorem style back for theorems defined in main file.
\theoremstyle{plain}

%\newenvironment{proof}{\noindent{\bf Proof}\hspace*{1em}}{\qed\bigskip}
\newenvironment{proof-sketch}{\noindent{\bf Sketch of Proof}\hspace*{1em}}{\qed\bigskip}
\newenvironment{proof-idea}{\noindent{\bf Proof Idea}\hspace*{1em}}{\qed\bigskip}
\newenvironment{proof-of-lemma}[1]{\noindent{\bf Proof of Lemma #1}\hspace*{1em}}{\qed\bigskip}
\newenvironment{proof-attempt}{\noindent{\bf Proof Attempt}\hspace*{1em}}{\qed\bigskip}
\newenvironment{proofof}[1]{\noindent{\bf Proof}
of #1:\hspace*{1em}}{\qed\bigskip}

%%%%%%%%%%%%%%%%%%%%%%%%%%%%%%%%%%%%%%%%%%%%%%%%%%%
% Useful Complexity Classes
%%%%%%%%%%%%%%%%%%%%%%%%%%%%%%%%%%%%%%%%%%%%%%%%%%%
\newcommand{\ntime}{\hbox{NTIME}}
\newcommand{\nspace}{\hbox{NSPACE}}
\newcommand{\conspace}{\hbox{co-NSPACE}}
\newcommand{\np}{\hbox{NP}}
\newcommand{\pspace}{\hbox{PSPACE}}
\newcommand{\lspace}{\hbox{L}}
\newcommand{\conp}{\hbox{coNP}}
\newcommand{\exptime}{\hbox{EXPTIME}}
\newcommand{\elem}{\hbox{E}}
\newcommand{\nl}{\hbox{NL}}
\newcommand{\bpp}{\hbox{BPP}}
\newcommand{\nregexp}{\hbox{NREGEXP}}
\newcommand{\tqbf}{\hbox{TQBF}}
\newcommand{\threesat}{\hbox{3SAT}}
\newcommand{\cvp}{\hbox{CVP}}
\newcommand{\stconn}{\hbox{STCONN}}
\newcommand{\ispath}{\hbox{ISPATH}}

%\newcommand{\class}{\hbox{$\mathbb{C}$}} 
%\newcommand{\class}{\hbox{$\mathbf{C}$}} 

\newcommand{\lep}{\leq _{\hbox{P}}}
\newcommand{\lel}{\leq _{\hbox{L}}}
\newcommand{\aspace}[1]{{\rm ASPACE}(#1)}
\newcommand{\atime}[1]{{\rm ATIME}(#1)}
\newcommand{\dtime}[1]{{\rm DTIME}(#1)}
\newcommand{\spa}[1]{{\rm SPACE}(#1)}
\newcommand{\ti}[1]{{\rm TIME}(#1)}
\newcommand{\ap}{{\rm AP}}
\newcommand{\al}{{\rm AL}}


%%%%%%%%%%%%%%%%%%%%%%%%%%%%%%%%%%%%%%%%%%%%%%%%%%%
% Useful Macros
%%%%%%%%%%%%%%%%%%%%%%%%%%%%%%%%%%%%%%%%%%%%%%%%%%%
\renewcommand{\Pr}[1]{\mathify{\mbox{Pr}\left[#1\right]}}
\newcommand{\Exp}[1]{\mathify{\mbox{Exp}\left[#1\right]}}
\newcommand{\bigO}O
\newcommand{\set}[1]{\mathify{\left\{ #1 \right\}}}
\def\half{\frac{1}{2}}

\def\implies{\Rightarrow}
\def\prob#1#2{{\mathop{{\rm Prob}}_{#1}}\left[#2 \right]}
\def\var#1#2{{\mathop{{\rm Var}}_{#1}}[#2]}
\def\expec#1#2{{\mathop{{\rm E}}_{#1}}[#2]}
\def\sizeof#1{\left| #1\right|}
\def\setof#1{\left\{ #1\right\}  }

\newcommand{\F}{{\mathbb{F}}}
\newcommand{\Z}{{\mathbb{Z}}}
%\newcommand{\qed}{\rule{7pt}{7pt}}

% \makeatletter
% \@addtoreset{figure}{section}
% \@addtoreset{table}{section}
% \@addtoreset{equation}{section}
% \makeatother

\newcommand{\FOR}{{\bf for}}
\newcommand{\TO}{{\bf to}}
\newcommand{\DO}{{\bf do}}
\newcommand{\WHILE}{{\bf while}}
\newcommand{\AND}{{\bf and}}
\newcommand{\IF}{{\bf if}}
\newcommand{\THEN}{{\bf then}}
\newcommand{\ELSE}{{\bf else}}
\newcommand{\N}{\mathbb{N}}

% \renewcommand{\thefigure}{\thesection.\arabic{figure}}
% \renewcommand{\thetable}{\thesection.\arabic{table}}
% \renewcommand{\theequation}{\thesection.\arabic{equation}}

% Calligraphic letters
\newcommand{\calA}{{\cal A}}
\newcommand{\calB}{{\cal B}}
\newcommand{\calC}{{\cal C}}
\newcommand{\calD}{{\cal D}}
\newcommand{\calE}{{\cal E}}
\newcommand{\calF}{{\cal F}}
\newcommand{\calG}{{\cal G}}
\newcommand{\calH}{{\cal H}}
\newcommand{\calI}{{\cal I}}
\newcommand{\calJ}{{\cal J}}
\newcommand{\calK}{{\cal K}}
\newcommand{\calL}{{\cal L}}
\newcommand{\calM}{{\cal M}}
\newcommand{\calN}{{\cal N}}
\newcommand{\calO}{{\cal O}}
\newcommand{\calP}{{\cal P}}
\newcommand{\calQ}{{\cal Q}}
\newcommand{\calR}{{\cal R}}
\newcommand{\calS}{{\cal S}}
\newcommand{\calT}{{\cal T}}
\newcommand{\calU}{{\cal U}}
\newcommand{\calV}{{\cal V}}
\newcommand{\calW}{{\cal W}}
\newcommand{\calX}{{\cal X}}
\newcommand{\calY}{{\cal Y}}
\newcommand{\calZ}{{\cal Z}}


% Some macro's from Speilman's course.

\makeatletter
\def\fnum@figure{{\bf Figure \thefigure}}
\def\fnum@table{{\bf Table \thetable}}
\long\def\@mycaption#1[#2]#3{\addcontentsline{\csname
  ext@#1\endcsname}{#1}{\protect\numberline{\csname 
  the#1\endcsname}{\ignorespaces #2}}\par
  \begingroup
    \@parboxrestore
    \small
    \@makecaption{\csname fnum@#1\endcsname}{\ignorespaces #3}\par
  \endgroup}
\def\mycaption{\refstepcounter\@captype \@dblarg{\@mycaption\@captype}}
\makeatother

\newcommand{\figcaption}[1]{\mycaption[]{#1}}
\newcommand{\tabcaption}[1]{\mycaption[]{#1}}


%%%%%%%%%%%%%%%%%%%%%%%%%%%%%%%%%%%%%%%%%%%%%%%%%%%%%%%%%%%%%%%%%%%%%%
% Feel free to ignore the rest of this file.

%%%%%%%%%%%%%%%%%%%%%%%%%%%%%%%
% Margins and page indentations
% DO NOT EDIT
%%%%%%%%%%%%%%%%%%%%%%%%%%%%%%
\textwidth=6in
\oddsidemargin=0.25in
\evensidemargin=0.25in
\topmargin=-0.1in
\footskip=0.8in
\parindent=0.0cm
\parskip=0.3cm
\textheight=8.00in
\setcounter{tocdepth} {3}
\setcounter{secnumdepth} {2}
\sloppy


\newcommand{\handout}[5]{
   \renewcommand{\thepage}{#1-\arabic{page}}
   \noindent
   \begin{center}
   \framebox{
      \vbox{
    \hbox to 5.78in { {\bf CS6840: Advanced Complexity Theory} \hfill #2 }
       \vspace{4mm}
       \hbox to 5.78in { {\Large \hfill #5  \hfill} }
       \vspace{2mm}
       \hbox to 5.78in { {\em #3 \hfill #4} }
      }
   }
   \end{center}
   \vspace*{4mm}
}

\newcommand{\lecture}[5]{\handout{#1}{#3}{Lecturer:~#4}{Scribe: #5}{Lecture~#1~: #2}}

\newcommand{\lectureplan}[1]{{\sc Lecture Plan:}#1\\}
\newcommand{\theme}[1]{{\sc Theme:} #1\\}

%	\begin{document}
%
%	\lecture{LECTURE NO.}{TITLE OF SCRIBE}{DATE}{Jayalal Sarma M.N.}{YOUR NAME}
%	\theme{THEME FOCUS}
%	\lectureplan{A BRIEF DESCRIPTION OF THE LECTURE}
%	
%	\section{TOPIC 1}
%	...
%	\section{TOPIC 2}
% 	...
%	\end{document}
% If there is not title, leave it as {}
 

\newtheorem{theorem}{Theorem}
\newtheorem{corollary}[theorem]{Corollary}
\newtheorem{lemma}[theorem]{Lemma}
\newtheorem{observation}[theorem]{Observation}
\newtheorem{proposition}[theorem]{Proposition}
\newtheorem{claim}[theorem]{Claim}
\newtheorem{fact}[theorem]{Fact}
\newtheorem{example}[theorem]{Example}
\newtheorem{assumption}[theorem]{Assumption}

\theoremstyle{definition}
\newtheorem{definition}[theorem]{Definition}

\theoremstyle{remark}
\newtheorem{remark}[theorem]{Remark}

% Setting theorem style back for theorems defined in main file.
\theoremstyle{plain}

%\newenvironment{proof}{\noindent{\bf Proof}\hspace*{1em}}{\qed\bigskip}
\newenvironment{proof-sketch}{\noindent{\bf Sketch of Proof}\hspace*{1em}}{\qed\bigskip}
\newenvironment{proof-idea}{\noindent{\bf Proof Idea}\hspace*{1em}}{\qed\bigskip}
\newenvironment{proof-of-lemma}[1]{\noindent{\bf Proof of Lemma #1}\hspace*{1em}}{\qed\bigskip}
\newenvironment{proof-attempt}{\noindent{\bf Proof Attempt}\hspace*{1em}}{\qed\bigskip}
\newenvironment{proofof}[1]{\noindent{\bf Proof}
of #1:\hspace*{1em}}{\qed\bigskip}

%%%%%%%%%%%%%%%%%%%%%%%%%%%%%%%%%%%%%%%%%%%%%%%%%%%
% Useful Complexity Classes
%%%%%%%%%%%%%%%%%%%%%%%%%%%%%%%%%%%%%%%%%%%%%%%%%%%
\newcommand{\ntime}{\hbox{NTIME}}
\newcommand{\nspace}{\hbox{NSPACE}}
\newcommand{\conspace}{\hbox{co-NSPACE}}
\newcommand{\np}{\hbox{NP}}
\newcommand{\pspace}{\hbox{PSPACE}}
\newcommand{\lspace}{\hbox{L}}
\newcommand{\conp}{\hbox{coNP}}
\newcommand{\exptime}{\hbox{EXPTIME}}
\newcommand{\elem}{\hbox{E}}
\newcommand{\nl}{\hbox{NL}}
\newcommand{\bpp}{\hbox{BPP}}
\newcommand{\nregexp}{\hbox{NREGEXP}}
\newcommand{\tqbf}{\hbox{TQBF}}
\newcommand{\threesat}{\hbox{3SAT}}
\newcommand{\cvp}{\hbox{CVP}}
\newcommand{\stconn}{\hbox{STCONN}}
\newcommand{\ispath}{\hbox{ISPATH}}

%\newcommand{\class}{\hbox{$\mathbb{C}$}} 
%\newcommand{\class}{\hbox{$\mathbf{C}$}} 

\newcommand{\lep}{\leq _{\hbox{P}}}
\newcommand{\lel}{\leq _{\hbox{L}}}
\newcommand{\aspace}[1]{{\rm ASPACE}(#1)}
\newcommand{\atime}[1]{{\rm ATIME}(#1)}
\newcommand{\dtime}[1]{{\rm DTIME}(#1)}
\newcommand{\spa}[1]{{\rm SPACE}(#1)}
\newcommand{\ti}[1]{{\rm TIME}(#1)}
\newcommand{\ap}{{\rm AP}}
\newcommand{\al}{{\rm AL}}


%%%%%%%%%%%%%%%%%%%%%%%%%%%%%%%%%%%%%%%%%%%%%%%%%%%
% Useful Macros
%%%%%%%%%%%%%%%%%%%%%%%%%%%%%%%%%%%%%%%%%%%%%%%%%%%
\renewcommand{\Pr}[1]{\mathify{\mbox{Pr}\left[#1\right]}}
\newcommand{\Exp}[1]{\mathify{\mbox{Exp}\left[#1\right]}}
\newcommand{\bigO}O
\newcommand{\set}[1]{\mathify{\left\{ #1 \right\}}}
\def\half{\frac{1}{2}}

\def\implies{\Rightarrow}
\def\prob#1#2{{\mathop{{\rm Prob}}_{#1}}\left[#2 \right]}
\def\var#1#2{{\mathop{{\rm Var}}_{#1}}[#2]}
\def\expec#1#2{{\mathop{{\rm E}}_{#1}}[#2]}
\def\sizeof#1{\left| #1\right|}
\def\setof#1{\left\{ #1\right\}  }

\newcommand{\F}{{\mathbb{F}}}
\newcommand{\Z}{{\mathbb{Z}}}
%\newcommand{\qed}{\rule{7pt}{7pt}}

% \makeatletter
% \@addtoreset{figure}{section}
% \@addtoreset{table}{section}
% \@addtoreset{equation}{section}
% \makeatother

\newcommand{\FOR}{{\bf for}}
\newcommand{\TO}{{\bf to}}
\newcommand{\DO}{{\bf do}}
\newcommand{\WHILE}{{\bf while}}
\newcommand{\AND}{{\bf and}}
\newcommand{\IF}{{\bf if}}
\newcommand{\THEN}{{\bf then}}
\newcommand{\ELSE}{{\bf else}}
\newcommand{\N}{\mathbb{N}}

% \renewcommand{\thefigure}{\thesection.\arabic{figure}}
% \renewcommand{\thetable}{\thesection.\arabic{table}}
% \renewcommand{\theequation}{\thesection.\arabic{equation}}

% Calligraphic letters
\newcommand{\calA}{{\cal A}}
\newcommand{\calB}{{\cal B}}
\newcommand{\calC}{{\cal C}}
\newcommand{\calD}{{\cal D}}
\newcommand{\calE}{{\cal E}}
\newcommand{\calF}{{\cal F}}
\newcommand{\calG}{{\cal G}}
\newcommand{\calH}{{\cal H}}
\newcommand{\calI}{{\cal I}}
\newcommand{\calJ}{{\cal J}}
\newcommand{\calK}{{\cal K}}
\newcommand{\calL}{{\cal L}}
\newcommand{\calM}{{\cal M}}
\newcommand{\calN}{{\cal N}}
\newcommand{\calO}{{\cal O}}
\newcommand{\calP}{{\cal P}}
\newcommand{\calQ}{{\cal Q}}
\newcommand{\calR}{{\cal R}}
\newcommand{\calS}{{\cal S}}
\newcommand{\calT}{{\cal T}}
\newcommand{\calU}{{\cal U}}
\newcommand{\calV}{{\cal V}}
\newcommand{\calW}{{\cal W}}
\newcommand{\calX}{{\cal X}}
\newcommand{\calY}{{\cal Y}}
\newcommand{\calZ}{{\cal Z}}


% Some macro's from Speilman's course.

\makeatletter
\def\fnum@figure{{\bf Figure \thefigure}}
\def\fnum@table{{\bf Table \thetable}}
\long\def\@mycaption#1[#2]#3{\addcontentsline{\csname
  ext@#1\endcsname}{#1}{\protect\numberline{\csname 
  the#1\endcsname}{\ignorespaces #2}}\par
  \begingroup
    \@parboxrestore
    \small
    \@makecaption{\csname fnum@#1\endcsname}{\ignorespaces #3}\par
  \endgroup}
\def\mycaption{\refstepcounter\@captype \@dblarg{\@mycaption\@captype}}
\makeatother

\newcommand{\figcaption}[1]{\mycaption[]{#1}}
\newcommand{\tabcaption}[1]{\mycaption[]{#1}}


%%%%%%%%%%%%%%%%%%%%%%%%%%%%%%%%%%%%%%%%%%%%%%%%%%%%%%%%%%%%%%%%%%%%%%
% Feel free to ignore the rest of this file.

%%%%%%%%%%%%%%%%%%%%%%%%%%%%%%%
% Margins and page indentations
% DO NOT EDIT
%%%%%%%%%%%%%%%%%%%%%%%%%%%%%%
\textwidth=6in
\oddsidemargin=0.25in
\evensidemargin=0.25in
\topmargin=-0.1in
\footskip=0.8in
\parindent=0.0cm
\parskip=0.3cm
\textheight=8.00in
\setcounter{tocdepth} {3}
\setcounter{secnumdepth} {2}
\sloppy


\newcommand{\handout}[5]{
   \renewcommand{\thepage}{#1-\arabic{page}}
   \noindent
   \begin{center}
   \framebox{
      \vbox{
    \hbox to 5.78in { {\bf CS6840: Advanced Complexity Theory} \hfill #2 }
       \vspace{4mm}
       \hbox to 5.78in { {\Large \hfill #5  \hfill} }
       \vspace{2mm}
       \hbox to 5.78in { {\em #3 \hfill #4} }
      }
   }
   \end{center}
   \vspace*{4mm}
}

\newcommand{\lecture}[5]{\handout{#1}{#3}{Lecturer:~#4}{Scribe: #5}{Lecture~#1~: #2}}

\newcommand{\lectureplan}[1]{{\sc Lecture Plan:}#1\\}
\newcommand{\theme}[1]{{\sc Theme:} #1\\}

\begin{document}
\lecture{5}{\FP vs $\#\P$ question}{Jan 12, 2012}{ Jayalal Sarma M.N.}{Prasun Kumar}
\theme{Between $\P$ and $\PSPACE$.}
\lectureplan{\FP vs $\#\P$ question, a counter part in the decision world. The class PP. PP vs P is equivalent to FP vs $\#\P$.}

We know $(\#\P = \FP) \implies (\P = \NP)$\\
Is the converse also true, i.e., does $(\P = \NP) \implies (\#\P = \FP)$ ?
With this question we ended the last lecture. Interestingly, the status of this
question is still unknown.

In this lecture, we will define a new class of languages \PP(probabilistic polynomial)
in decisoon world and then try to answer the above question by considering the 
containment relationship between the classes \PP and \NP. 


\section{Complexity Class \PP}
With respect to a nondeterministic Turing machine(NTM), we have two different
classes of languages as follows, which differs in their acceptance conditions by NTM 

(i) \NP = $\{$L $|\exists$ NTM $M_1$, x $\in$ L $\iff M_1$ has atleast one accepting path on x $\}$\\
(ii)\coNP = $\{$L $|\exists$ NTM $M_2$, x $\in$ L $\iff$all paths in $M_2$ are accepting paths on x $\}$

So, in terms of number of accepting paths on input {\em x}, we are talking of
extremums. On one side, in \NP  we talk of atleast one accepting path, and on
the other side, in \coNP  we talk of all accepting paths. So here we can ask a
question that will we get a different class of languages if we have larger than some fraction 
of total number of paths as accepting paths by a nondeterministic turing machine
on input {\em x}. The answer to this question is affirmative if we have this fraction
to be half and the corresponding class of languages is called \PP (probabilistic polynomial)
which is defined as follows \\ 

\PP =$\{$L $|\exists$ NTM $M$, x $\in$ L $\iff$ atleast half of the total number of paths are accepting paths on x $\}$

\begin{claim}
 \coNP $\subseteq$ \PP
\end{claim}
\begin{proof}
 directly follows from the definition.
\end{proof}

\begin{claim}
  \NP $\subseteq$ \PP
\end{claim}
\begin{proof}
Let L $\in$  \NP  via a nondeterministic turing machine M,  x $\in$ L $\iff$M has atleast one accepting path on x. 
Our aim is to give another turing machine $M^{'}$ for the same language L such that
x $\in$ L $\iff M^{'}$ has more than half of the total number of paths as accepting paths on x.\\
Description of $M^{'}$ :\\ (i)  simulate M on {\em x}\\
                        (ii) if M accepts then choose one bit nondeterministically and accept in both branches.\\
                       (iii)if M rejects then choose one bit nondeterministically,accept in one branch
                            and reject in other.\\
Assume the height of computation tree of M on {\em x} is $p(n)$. Now, whenever x $\in$ L, number of
accepting paths of $M^{'}$ on {\em x} is $\ge 2^{p(n)}$ where total number of paths in computation tree
for $M^{'}$ is $2^{p(n)+1}$. Whenever x $\not\in$ L,all paths reject in M thus  number of accepting and rejecting
paths in computation tree for $M^{'}$ are same and is equal to $2^{p(n)}$.
\end{proof}
Now, we have the following containment relationship among different classes of languages
 
\begin{center}
\PSPACE \\ 
$\vert$ \\
\PP \\
$\vert$ \\
\NP \\
$\vert$ \\
\P
\end{center}

\begin{theorem}
 $(\#\P = \FP) \iff (\PP = \P)$
\end{theorem}
\begin{proof}
($\Rightarrow$) Assume $\#\P = \FP$, our aim is to prove $\PP = \P$. Equivalently we have to prove that\\
(i) $\P \subseteq \PP$ : it directly follows from the basic containment relation.\\
(ii)$\PP \subseteq \P$ : so we need to show that $(L \in \PP) \implies (L \in \P)$.\\
Let $L \in \PP$ via a NTM M $\implies x \in L$ if and only if more than half of the paths are accepting
paths. Define $\# accept_M(x)$ as the {\em number of accepting paths of M on x}. Now let $f(x)=\# accept_M(x)$
on any input x then $f \in \#p$(by definition). So, $f \in \FP$(by our assumption) thus we can write $f(x)$
in poly time and by checking its value in poly time we can decide whether $x \in L$ or not. So, $L \in \P$.\\

($\Leftarrow$)Assume $\PP = \P$, our aim is to prove $\#\P = \FP$. Equivalently we have to prove that\\
(i) $\FP \subseteq \#\P$ : it directly follows from the basic containment relation.\\
(ii)$\#\P \subseteq \FP$ : so we need to show that $(f \in \#\P) \implies (f \in \FP)$.\\
Let $f\in \#\P$ via a NTM $M$ such that $\forall x$ $f(x)=\# accept_M(x)$. The naive approach
to compute $f(x)$ is to compute the number of accepting paths in computation tree of height $p(n)$ which 
will take exponential time. So we use an alternative way to show that $f \in \FP$ by finding 
the minimum value of $k$ such that $k+\# accept_M(x)\textgreater 2^{p(n)}$. Our aim has now shifted to 
given a $k$,how to figure out $k+\# accept_M(x)\textgreater 2^{p(n)}$. For this, we construct an another 
nondeterministic turing machine $N$ as follows\\
(a){\em Construct a NTM $M^{'}$ having exactly $k$ number of accepting states.}\\

We can always construct such $M^{'}$ by making first lexicographic $k$ paths as accepting paths
and make remaining as rejecting paths.\\

(b){\em Extend the depth of $M^{'}$ from $\log(k)$ to $p(n)$ keeping the number of accepting paths same.}\\

This we can do by making the computation tree a full and complete binary tree of depth $p(n)$. In the
process, for all accepting paths with depth $\le p(n)$ we introduce a subtree at the corresponding leaf 
such that we get a full and complete binary tree and in all those subtrees make the lexicographic first
(leftmost) path as accepting path and make remaining as rejacting paths. Similarly,for all rejecting paths
with depth $\le p(n)$ make all the paths in the subtree as rejecting paths.\

(c)Introduce a new root node which goes to root of computation tree of $M$ on input $0$ and on input $1$
it goes to root of computation tree of $M^{'}$.\
 
Thus it runs in $p(n)+1$ depth and $\# accept_N(x)=k+\# accept_M(x)$.
%Now $\# accept_N(x)= k+\# accept_M(x)\ge 2^{p(n)}$=1\2* $ 2^{p(n)+1}$=(1\2)*{\em total number of paths in
%computation tree of N on x}. So, $x\in L(N)\iff \# accept_N(x)\ge2^{p(n)$. Thus,$L(N)\in \PP\implies L(N)\in \P$
(by our assumption). We are interested in minimum $K$ and $0\le K \le 2^{p(n)}$, so we can make at most $p(n)$
calls to find minimum $K$. So, we can write down $f(x)$ in poly time , hence $f\in\FP$.

\end{proof}



\end{document}
