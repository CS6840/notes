\Lecture{Dinesh K.}{Jan 16, 2012}{7}{A Combinatorial Interpretation of Permanent}
%\theme{Between $\P$ and $\PSPACE$.}
%\lectureplan{Combinatorial interpretation of permanents over integer
%matrices, using cycle cover}

\newcommand{\countervalue}[1]{\arabic{#1} \addtocounter{#1}{1}}
\newcommand{\perm}{{\sf perm}}
We consider the problem of computing the permanent of a square matrix $A_{n
\times n}$ defined as
\[ \perm(A) = \sum_{\substack{\sigma \in S_n}} \prod_{\substack{i = n}}^n
A_{i, \sigma(i)} \]
where $S_n$ denotes the set of all permutations of $\left \{1, 2, \ldots, n
\right \}$. Note that this formula is very similar to that of the determinant
computation which can be expressed as,
\[ |A| = \sum_{\substack{\sigma \in S_n}} sign(\sigma) \prod_{\substack{i = n}}^n
A_{i, \sigma(i)} \]
where $sign$ of a permutation is $-1/+1$ depending on if the number of inversions
in the permutations is odd or even. 
Though the definitions are similar "syntactically", the computation problems are very different
in the level of hardness. In this lecture, we try to understand more on
permanents and will come up with equivalent combinatorial characterisations.

\section{Characterisation of Permanents of Binary Square Matrices}
Our aim is to how characterise the problem of computing permanent as a
combinatorial problem thereby gives us a handle for attacking the problem. 

Consider a bipartite graph $G(X, Y, E)$ where $|X|=|Y| = n$. Consider the
adjacency matrix $A=(A_{i,j})_{n \times n}$ where,
\[ A_{i,j} = \left \{
	\begin{array}{rl}
	 1 & \text{if } (x_i,y_j) \in E, x_i \in X, y_j \in Y \\
	 0 & \text{otherwise}
	\end{array} \right . \]
We give the following characterisation for permanent of $A$.
\begin{lemma}
$\perm(A) = \left | \text{\# of perfect matchings in } G \right |$
\end{lemma}
\begin{proof}
We show that every permutation $\sigma \in S_n$ corresponds to a unique
perfect matching $\calM$.

($\Rightarrow$) Note that every permutation $\sigma$ of $S_n$ with the property
$\prod_{\substack{i=1}}^n A_{i, \sigma(i)} = 1$ means that there is an edge
connecting vertices $x_i$ and $y_{\sigma(i)}$. Since $\sigma$ is a permutation,
the map is bijective and $y_{\sigma(i)}$ exists for every $x_i$ and is
unique. Hence by definition $\{(x_i, y_{\sigma(i)})| i \in \{1, 2, \ldots,
n\}\}$ forms a perfect matching. 

($\Leftarrow$) Similarly, given a perfect matching $\calM$,
one can define a permutation from $X$ to $Y$ as $\sigma(i)=j$ if $(x_i, y_j)
\in \calM$. It can be seen that the map is bijective and $\prod_{i=1}^n A_{i,
\sigma(i)}$ will be $1$. 

Hence for every $\sigma \in S_n$,
\begin{equation} 
\label{eq:arg}
\prod_{\substack{i=1}}^n A_{i,\sigma(i)} = 1 \iff \left \{ (x_i,
y_{\sigma(i)}) | i \in \{1, 2, \ldots, n \} \text{ is a matching} \right \} 
\end{equation}
Now taking sum over all permutations (on both sides) would give the lemma.
\end{proof}

Hence for a $\{0,1\}$ square matrix, checking if \perm~ is larger than $0$ can
be solved by showing a perfect matching in the corresponding graph.
Standard flow algorithms like the \emph{Fork-Fulkerson algorithm} can be used to find the 
matching in polynomial time.

This characterisation also gives us a \#\P~machine for \perm, thereby showing 
that \perm~is in \#\P.
\begin{lemma}
For $A \in \{0,1\}^{n \times n}$, $\perm(A) \in \#\P$
\end{lemma}
\begin{proof}
Following machine $M$ computes \perm(A).\\
$M$ = ``On input $A_{n\times n}$ 
\begin{enumerate}
\item Construct a bipartite graph $G(X, Y, E)$ with  
$X=\{x_1,x_2,\ldots,x_n\}, \\ Y=\{y_1,y_2,\ldots,y_n\}, (x_i,y_j) \in 
E \iff A_{i,j}=1$
\item Non deterministically guess a subset of edges.
\item Check if they form a perfect matching and accept iff the edges form a
perfect matching."
\end{enumerate}

Note that $M$ runs in polynomial time and by the equivalence~(\ref{eq:arg}),
each perfect matching contributes a count of one to the permanent. Hence the
number of accepting paths of $M$ will exactly be equal to the \perm(A).
\end{proof}

\section{Characterisation of Permanents of Integer matrices}

Now we get to matrices with integer entries. Since the entries could also be negative, the permanent of the matrix 
can be negative. Hence this function cannot be in $\#\P$, but we will show that it is in $\FP^{\#\P}$. 
To do this, we start with the combinatorial characterisation of the permanent, for which we need the notion of cycle covers.

%\subsection{Cycle covers}
\begin{definition}(Cycle Cover) 
Consider a directed weighted graph $G(V,E,w)$ with $w : E
\to \N$. A \emph{cycle cover} of $G$ is a subset of edges that forms a set of vertex disjoint directed
cycles in $G$ such that each vertex is a part of at least one cycle in the cycle cover.

\emph{Weight of a cycle cover} $\calC$ is defined as the product of weights of
edges in $\calC$.
\[ wt(\calC) = \prod_{\substack{e \in \calC}} w(e) \]
\end{definition}

Consider the following examples. Note the effect of adding self loops.
\begin{figure}[htp!]
\centering
\begin{tikzpicture}
\tikzset{
    myarrow/.style={->, >=latex', thick},
}
\newcounter{vertid}
\setcounter{vertid}{1}
\foreach \x in {0,3}
{
	\draw [myarrow] (\x+1,0) node[above]{\countervalue{vertid}}--(\x+0,1);
	\draw [myarrow] (\x+0,1) node[above]{\countervalue{vertid}}--(\x+1,2);
	\draw [myarrow] (\x+1,2) node[above]{\countervalue{vertid}}--(\x+2,1);
	\draw [myarrow] (\x+2,1) node[above]{\countervalue{vertid}}--(\x+1,0);
};
\draw[myarrow] (2,1) -> (3,1);
\end{tikzpicture}
\quad
\begin{tikzpicture}
\tikzset{
    myarrow/.style={->, >=latex', thick},
}
%\newcounter{vertid}
\setcounter{vertid}{1}
	\draw [myarrow] (1,0) node[above]{\countervalue{vertid}}
	--(0,1.38);
	\draw [myarrow] (0,1.38) node[above]{\countervalue{vertid}}
	--(1,2.76);
	\draw [myarrow] (1,2.76) node[above]{\countervalue{vertid}}
	--(2.62,2.2);
	\draw [myarrow] (2.6,2.2) node[above]{\countervalue{vertid}}
	--(2.6,0.5);
	\draw [myarrow] (2.6,0.5) node[above]{\countervalue{vertid}}
	--(1,0);

\foreach \x in {6.24}
{
	\draw [myarrow] (\x-0,1.38) node[above]{\countervalue{vertid}}
	--(\x-1,0);
	\draw [myarrow] (\x-1,2.76) node[above]{\countervalue{vertid}}
	--(\x-0,1.38);
	\draw [myarrow] (\x-2.62,2.2) node[above]{\countervalue{vertid}}
	--(\x-1,2.76);
	\draw [myarrow] (\x-2.6,0.5) node[above]{\countervalue{vertid}}
	--(\x-2.6,2.2);
	\draw [myarrow] (\x-1,0) node[above]{\countervalue{vertid}}
	--(\x-2.6,0.5);

	\draw [myarrow] (2.6,0.5) -- (\x-2.6,0.5);
	\draw [myarrow] (\x-2.6,2.2) -- (2.6,2.2);

};
\end{tikzpicture}

\centering
\begin{tikzpicture}
\tikzset{
    myarrow/.style={->, >=latex', thick},
}
%\newcounter{vertid}
\setcounter{vertid}{1}
	\draw [myarrow] (1,0) node[above] (A) {\countervalue{vertid}}
	--(0,1.38);
	\draw [myarrow] (0,1.38) node[above] (B) {\countervalue{vertid}}
	--(1,2.76);
	\draw [myarrow] (1,2.76) node[above] (C){\countervalue{vertid}}
	--(2.62,2.2);
	\draw [myarrow] (2.6,2.2) node[above]{\countervalue{vertid}}
	--(2.6,0.5);
	\draw [myarrow] (2.6,0.5) node[above]{\countervalue{vertid}}
	--(1,0);

	\path [myarrow] (A) edge [loop below] (A);
	\path [myarrow] (B) edge [loop left] (B);
	\path [myarrow] (C) edge [loop above] (C);

\foreach \x in {6.24}
{
	\draw [myarrow] (\x-0,1.38) node[above] (E) {\countervalue{vertid}}
	--(\x-1,0);
	\draw [myarrow] (\x-1,2.76) node[above](F){\countervalue{vertid}}
	--(\x-0,1.38);
	\draw [myarrow] (\x-2.62,2.2) node[above]{\countervalue{vertid}}
	--(\x-1,2.76);
	\draw [myarrow] (\x-2.6,0.5) node[above] {\countervalue{vertid}}
	--(\x-2.6,2.2);
	\draw [myarrow] (\x-1,0) node[above](D) {\countervalue{vertid}}
	--(\x-2.6,0.5);

	\draw [myarrow] (2.6,0.5) -- (\x-2.6,0.5);
	\draw [myarrow] (\x-2.6,2.2) -- (2.6,2.2);

	\path [myarrow] (D) edge [loop below] (D);
	\path [myarrow] (E) edge [loop right] (E);
	\path [myarrow] (F) edge [loop above] (F);
}
\end{tikzpicture}
\caption{{\small In the first example, we can see that the only cycle cover is 
\texttt{(1,2,3,4),(5,6,7,8)}. For the second one also there is only one
cycle cover, namely \texttt{(1,2,3,4,5),(10,9,8,7,6)}.
The third is a modified version of the second example. Now there are
\emph{two} cycle covers namely, \texttt{\{(1,2,3,4,5),(10,9,8,7,6)\},
\{(1),(2),(3),(6),(7),(10),(4,5,9,8)\}}}.
}
\end{figure}

We are now set to give a characterisation of permanent over matrices in 
$\Z^{n \times n}$, which we will be crucially using in the next lecture.

\begin{lemma}
Let $A \in \Z^{n \times n}$. Let $G$ be the directed weighted graph obtained
by interpreting $A$ as the weighted adjacency matrix of $G$, i,e. 
\[|V(G)| = n,\] \[ \forall 1 \le i,j \le n, wt(i, j) = w \iff A_{i,j} = w \]
then
\[ \perm(A) = \sum_{\substack{\calC \in Cyclecover(G)}} wt(\calC) \]
\end{lemma}
\begin{proof}
We show that corresponding to every permutation $\sigma \in S_n$, there is a
unique cycle cover $\calC$. That is,
\[ \prod_{\substack{i=1}}^n A_{i,\sigma(i)} = k \iff \exists~\calC \text{ such
that } wt(\calC) = k \]

($\Leftarrow$) Suppose there is a cycle cover $\calC$ such 
that $wt(\calC) = k$. Without loss of generality assume $k \not = 0$. 
(If $k=0$, then one of the edges is not  present and hence no cycle cover is 
possible using the edges in $\calC$).  
Now we can define a permutation to be $\sigma(i) = j~\forall (i,j)\in \calC$.
Since $\calC$ covers all vertices, $\sigma$ is defined for all $\{1,2,\ldots,
n\}$ and since $\calC$ is composed of cycles, every $(i,j)$ will be unique.
Hence the map $\sigma$ will be bijective and is a valid permutation in $S_n$.
Also $\prod_{i=1}^n A_{i,\sigma(i)} = \prod_{(i,j) \in \calC} wt(i,j) =
wt(\calC) = k$.

($\Rightarrow$) Similarly, given a permutation $\sigma$ such that $\prod_{i=1}^n
A_{i,\sigma(i)} = k$, we construct a cycle cover as follows.

If $k=0$, then there is no cycle cover possible with the given permutation
since a zero weight edge is appearing. Hence let $k \not = 0$. Fix an $i\in
\{1,2,\ldots,n\}$. Consider the sequence $(i, \sigma(i), \sigma^2(i), \ldots,
\sigma^n(i))$ where $\sigma^i$ for $i \ge 0$ is obtained by applying $\sigma$
function $i$ times. It can be seen by pigeon hole principle that atleast one
values in the sequence must repeat (since, $\sigma$ is defined on $n$ 
terms and there are $n+1$ terms in the sequence). Without loss of generality, 
let $\sigma^l(i)$ and $\sigma^m(i)$ repeat with $ 0 \le l < m \le n$. 
Since all the weights  are non-zero, the vertices 
$\sigma^l(i), \sigma^{l+1}(i), \ldots, \sigma^m(i) = \sigma^l(i)$ have 
edges between them and clearly forms a cycle. We can find
all the cycles by repeating this process for various values of $i$. 
Note that since $\sigma$ is a permutation, the cycles obtained will be disjoint 
and will cover the entire graph.

Now taking sum over all permutations on both sides proves the claim.
\end{proof}

Using this result the following observation (which is left as an exercise) can
also be made.
\begin{observation}
For $A \in \Z^{n\times n}$ computing permanent of $A$ is in $\FP^{\#\P}$.
\end{observation}
