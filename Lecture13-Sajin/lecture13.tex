\Lecture{Sajin Koroth}{January 28, 2012}{13}{Self Reducibility of $\SAT$, Complete problem for  $\Sigma_k^\P$}
%\theme{Polynomial Hierarchy and its relation to $\P/\poly$}
%\lectureplan{
Recall that we introduced the advice based class $\P/\poly$
in the last lecture. We also saw that $\BPP \subsetneq \P/\poly$, and
by definition $\P \subsetneq \P/\poly$. But we don't know whether
$\NP\subsetneq \P/\poly$ or not. Hence if we could prove that $\NP
\not\subset \P/\poly$ then we would essentially be separating $\P$
from $\NP$. The reason why most of the complexity theorists believe $\NP
\not\subset \P/\poly$ is, if $\NP \subset \P/\poly$ then we would be
able to prove that $\PH=\Sigma_2^\P$, contrary to the common belief
that $\PH$ does not collapse. In today's lecture we will detail two
key ingredients needed for showing the above mentioned conditional
collapse of $\PH$, \textbf{a complete problem for $\Sigma_k^\P$} and
\textbf{self reducibility property of $\SAT$}
 % }

\section{Complete problem for the hierarchy}

We will first show a complete problem for the $k$th level of
polynomial hierarchy. Later on we will use the self-reducibility
nature of this problem to show the conditional collapse mentioned
earlier. Recall that a language $L$ is said to be in $\Sigma_k^\P$ 
if there exists polynomials $p_1,\dots,p_k$ and a machine $M$ running
in deterministic polynomial time such that
\begin{displaymath}
  x \in L \iff \exists y_1 \forall y_2 \exists y_3 \dots Q_k y_k
  \left[ M(x,y_1,y_2,y_3,\dots,y_k) = 1\right] , \forall i,|y_i|\leq p_i(|x|)
\end{displaymath}

Cook-Levin theorem guarantees that machine $M$ on input $x$ can be
converted into formula $\phi_x$ in polynomial time on variables
$y_1,y_2,\dots,y_k$ such that $\phi_x(y_1,y_2,y_3,\dots,y_k)$ is
satisfiable if and only if $M(x,y_1,y_2,y_3,\dots,y_k)$ accepts. Hence
we can say that the following problem is complete for $\Sigma_k^\P$,
\begin{definition}[$\Sigma_k-\SAT$] 
$\Sigma_k-\SAT$ is the set of all quantified Boolean formulas with at
most $k$ alternations (starting with an existential quantifier) which
are true. That is
\begin{displaymath}
  \Sigma_k-\SAT = \left\{ \exists y_1 \forall y_2 \exists y_3 \dots Q_k y_k
  \phi(y_1,\dots,y_k) \mid \exists y_1 \forall y_2 \exists y_3 \dots Q_k y_k
  \phi(y_1,\dots,y_k) \text{ is true} \right\}
\end{displaymath}
\end{definition}

The above problem is clearly in $\Sigma_k^\P$ as you can in polynomial
time construct from a formula, a machine in $\P$ for checking if the
formula is satisfiable or not given an assignment of all the variables
as input. The problem is $\Sigma_k^\P$ hard because
of Cook-Levin reduction from any machine in $\P$ to an equivalent formula.
 
\section{Self reducibility of $\SAT$}

Suppose we are given that $\NP \subset \P/\poly$ then we know that
there is a polynomial time deterministic Turing machine and a
polynomial length advice string for each input length such that the
machine decides a given language in $\NP$. We will sketch how this can
cause a collapse in the Polynomial Hierarchy, without giving the
details but exposing some difficulties which we have to overcome
before getting to the proof. To prove that $\PH$ collapses to
$\Sigma_2^\P$ it suffices to show that $\Sigma_3^\P=\Sigma_2^\P$. Recall
that $\Sigma_3^\P$ is the set of true quantified Boolean formulas which
are of the form $\exists y_1 \forall y_2 \exists y_3
M(x,y_1,y_2,y_3)$, and $\Sigma_2^\P$ are true quantified Boolean
formulas which are of the form $\exists y_1 \forall y_2
M(x,y_1,y_2)$. Also we are given that $\NP \subset \P/\poly$ hence for
any $L\in \NP$ there exists $h:N \to \{0,1\}^*$ and an $M\in \P$ such
that $x \in L$ if and only if $(x,h(|x|))$ is accepted by $M$. The
idea to place $\Sigma_3^\P$ in $\Sigma_2^\P$ is the following, the third
there exists $y_3$ and $M(x,y_1,y_2,y_3)$ can be combined to a machine
in $\NP$, where it first guesses a string $y_3$ of size $p_3(|y_3|)$
and then runs $M$ on $(x,y_1,y_2,y_3)$. We have assumed that
equivalent to this $\NP$ machine there is a $\P/\poly$ machine, and
even though we don't know the advice string we know there exists a
good advice string, and given the advice string the last ``there
exists'' quantifier in $\Sigma_3^\P$ can be eliminated by replacing it
with the polynomial time machine which is given the advice string,
hence we would a get a language in $\Sigma_2^\P$. But unfortunately we
don't know the advice string, hence the next best thing to do is to
guess the advice string using the first ``there exists'' quantifier in
$\Sigma_2^\P$. We are guaranteed that at least one guess is the
correct advice string. But there is a catch here, we could have
guessed the advice string incorrectly in some branch which in turn
could have led the machine $M$ to accept incorrectly thus falsely
accepting a string outside the language $L$ in $\Sigma_3^\P$. To get
around this problem we will use the first part to reduce the problem
in $\Sigma_3^\P$ to $\Sigma_3-\SAT$ and then use an algorithm for
$\SAT$ which given a sub-routine which tells a formula is satisfiable
or not, which uses a crucial property of the $\SAT$ problem,
self-reducibility to construct a satisfying assignment for the given
formula. And in the case it cannot construct a satisfying assignment
we would be able to guarantee that the advice string guessed is bad.

Self reducibility of $\SAT$ refers to the property of the $\SAT$
problem that checking a formula on $n$ variables is satisfiable
reduces to checking the satisfiability of two formulas on $n-1$
variables. This property leads to a polynomial time algorithm for
constructing a satisfying assignment given a polynomial time
sub-routine deciding the decision version of $\SAT$ problem
correctly. Algorithm~\ref{SAT_ASGN_ALGO} constructs a satisfying
assignment given a sub-routine which correctly solves $\SAT$ instances
of up to $n$ variables. Notice that one important property of
Algorithm~\ref{SAT_ASGN_ALGO} is that even if the sub-routine which
checks the satisfiability of a formula is wrong, the algorithm would
not be accepting an un-satisfiable formula as a satisfiable
formula. Because at the end of the algorithm we are checking whether
the assignment constructed by the algorithm is satisfiable or not, so
even if the sub-routine for $\SAT$, \textbf{SATISFIABLE} is
erroneous we would not be able to construct a satisfying assignment
for an un-satisfiable formula. But it might fail to construct a
satisfying assignment for a satisfiable formula if the sub-routine is
erroneous.

% Could not be added due to conflict with clrs package. Needs to be rewritten
% in algorithm 
%\begin{codebox}
%\label{SAT_ASGN_ALGO}
%  \Procname{$\proc{SATISFYING-ASSIGNMENT}(\phi(x_1,\dots,x_n))$}
%%  \li $\id{\psi(x_1,\dots,x_n)} \gets \phi(x_1,\dots,x_n)$
%  \li \For $i \gets 1$ \To $\id{n}$
%  \li \Do
%  \li $\id{\psi^{'}} \gets \psi(x_i=0)$
%  \li $\id{\psi^{''}} \gets \psi(x_i=1)$
%  \li \If $\proc(SATISFIABLE(\psi^{'}))$
%  \li \Then $\id{a_i} \gets 0$
%  \li       $\id{\psi} \gets \psi(0,x_2,\dots,x_n)$
%  \li \ElseIf $\proc(SATISFIABLE(\psi^{''}))$
%  \li \Then $\id{a_i} \gets 1$
%  \li       $\id{\psi} \gets \psi(1,x_2,\dots,x_n)$
%  \li \ElseNoIf
%  \li \Return $\const{impossible}$
%      \End
%  \li \If $\phi(a_1,\dots,a_n) = 1$ 
%  \li \Then \Return $(a_1,\dots,a_n)$
%  \li \Else \Return $\const{impossible}$
%      \End
%      \End
%\end{codebox}

