\documentclass[11pt]{article}
\usepackage{amsmath,amssymb,amsfonts,amsthm,complexity}

\newcommand{\perm}{{\sf perm}}
\newcommand{\GNI}{{\sf GNI}}

\input{preamble.tex}
\begin{document}
\lecture{20}{Interactive protocol for Permanent}{Feb 13, 2012}{Jayalal Sarma M.N.}{Anup Joshi}
\theme{Between $\P$ and $\PSPACE$.}
\lectureplan{ Proof of correctness of $\GNI$ protocol, Historical Aspects of $\IP$, $P^{\#P}$, and
final proof. The interactive protocol for the permanent (outline).}

\section{Proof of correctness of $\GNI$ protocol}
In the last lecture we have seen an interactive protocol for $\GNI$, here we argue about the
correctness of the interactive protocol. If $G_1$ and $G_2$ are the two graphs for which we want to
check if they are isomorphic or not, lets assume that both of them are not isomorphic, i.e.
\begin{tabbing}
$G_1 \ncong G_2$, \=$\Rightarrow \nexists \tau : G_1 \rightarrow G_2$ \\ 
\>  $\Rightarrow \exists! b'$ such that $\sigma(G_{b'}) \cong G'$ \\
\>  $\Rightarrow$ Return $b'$
\end{tabbing}
Now, lets consider the case when $G_1$ and $G_2$ are isomorphic, i.e.
\begin{tabbing}
$G_1 \cong G_2$, \=$\Rightarrow \exists \tau : G_1 \rightarrow G_2$ \\ 
\>  $\Rightarrow G' \cong \tau(G_1)$ \\
\>  $\Rightarrow G' \cong \tau(G_2)$ \\
\end{tabbing}
Since it is isomorphic, $b'$ can be any of $1$ or $2$. Hence, with probability $1/2$, prover can
convince the verifier by replying randomly. Here, in order to get a very small error probability
say $\epsilon$, we can repeat the process on the same input, and take the majority vote. Hence, in
summary we have the following situation:

\begin{tabbing}
If \=$(G_1, G_2) \in GNI \Rightarrow $ Pr[Verifier accepts] = $1$ \\ 
\>  $(G_1, G_2) \notin GNI \Rightarrow $ Pr[Verifier accepts] $\leq 1/2$ \\
\end{tabbing}

\section{$\NP$ versus $\IP$}
In $\NP$, the verifier deterministically verifies the guessed path. We also call this guessed path
or string as the $certificate$ if the computation is accepted on that path. 

In $\IP$, however, the verifier is non-deterministic. The proofs are big, and the only way to solve
it is to pick random places in the proof, and run multiple protocols on them. We can say that $\IP$
is the randomized relaxation of the class $\NP$.

\section{Historical aspects of $\IP$}
The class $\IP$ has an interesting history behind it. On November 27, 1989, Noam Nisan proved a
weaker result showing that $Perm \in \IP$. He sent this result written on an e-mail to several of
his colleagues. From then on all the people in the mailing-list started their own research on the
topic, and in just 2 weeks, i.e. on December 13, 1989, Lance Fortnow sent an e-mail to all the
mailing-list members that $P^{\#\P}$ was in $\IP$. And again in just 2 weeks time, i.e. on December
26, 1989, Adi Shamir announced his findings. He showed that $\PSPACE \subseteq \IP$, and thus
$\PSPACE = \IP$. The reverse containment, i.e., $\IP \subseteq \PSPACE$ was showed earlier by
Papadimitriou in 1987.

\begin{theorem}
 $\P^{\#\P} \subseteq \IP$
\end{theorem}
\begin{proof}
It is enough if we show that the permanent computation can be done by a prover-verifier pair. This
proof was left as an exercise. 
\end{proof}

\section{Interactive protocol for Permanent}
Lets recall that the permanent $\perm(M)$ of an $n \times n$ matrix $M = (m_{ij})$ is defined
as 
\begin{center}
 \begin{math}
  \perm(M) = \Sigma_\sigma \Pi^{n}_{i=1} m_{i, \sigma(i)}.
 \end{math}
\end{center}
Obtaining the permanent of a matrix is not easy to compute, but we have an interactive protocol for
Permanent. Let \~{M}$_{ij}$ be the minor of $M$ obtained by deleting $i^{th}$ row and $j^{th}$
column. The protocol is as given below:

Protocol:
\begin{enumerate}
 \item Prover commit on a value of $\perm(M)$.
 \item Verifier asks for permanent of $q_j = $\~{M}$_{ij}$, for $1 \leq i \leq n$.
 \item Prover provides permanent
 \item Verifier checks if $\Sigma_{j = 1}^{n} q_j M[i, j] = q$. If yes, repeat step $2$, else
$reject$
\end{enumerate}

According to the protocol, the prover initiates the task by first committing on a permanent of a $n
\times n$ matrix. After that a sequence of interaction between the prover and verifier follows in
which the prover provides the permanent of smaller matrices. The verifier verifies that the
permanent of the smaller matrices are indeed correct by going through some consistency checks on the
received values. If the prover has not cheated even once, then the consistency checks are always
satisfied, and eventually it is accepted. If the prover cheats then it has to cheat only once to
fool the verifier. If the verifier catches the cheating prover it rejects, and if the prover was
honest then the verifier accepts. As we see, this protocol requires exponential time to check the
values received from the prover. There will be $n(n-1)...(n - n + 1) = n!$ checks needed, which the
verifier cannot do. 

Since the verifier cannot check all $q_j$, instead of this we randomly pick a $q_j$ and verify it
recursively. But even in this method the probability of catching a cheating prover is very low,
since the prover has to cheat only once. A solution is to consider the matrix $D(x) = xA_1 + (1 -
x)A_2$, where $A_1$ and $A_2$ are two submatrices, and there permanent is $p_1$ and $p_2$
respectively. Here we observe that $D(1) = A_1$ and $D(0) = A_2$. Let $f(x) = Perm(D(x))$, then
$f(0) = Perm(D(0)) = Perm(A_2) = p_2$, and $f(1) = Perm(D(1)) = Perm(A_1) = p_1$. We note that the
$Perm(D(x))$ is a polynomial of degree $\leq n$, since the entries of $D(x)$ are linear functions
of $x$. The prover has to cheat on one of $A_1$ or $A_2$, in order to meet the consistency check of
$D(0) = A_2$ and $D(1) = A_1$, hence once the prover has cheated it has to repeatedly cheat on the
subsequent interactions in order to fool the verifier. This leaves a very low margin of error for
the prover. This is the LFKN protocol, which we will see in detail in the next lecture.
\end{document}