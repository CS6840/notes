\documentclass[11pt]{article}
\input{preamble.tex}

\begin{document}
\lecture{No. 27}{Inapproximability}{Feb 21, 2012}{Jayalal Sarma M N}{Balagopal}
\theme{Inapproximability}
\lectureplan{Inapproximability of Independent set Problem. GAPCSP to GAPIS, 
PCP for LIN, Attempts, Proof in the long-code form. Need of linearity testing.}


Our aim is to show the inapproximability of \lang{MAXINDSET}. For this 
purpose, we introduce the problem \lang{GAPINDSET(s,c)}. An instance 
of \lang{GAPINDSET(s,c)} is a graph $G$ which is guaranteed to either 
have an independent set of size at least $cn$ or to have no independent 
set of size $sn$ (i.e., all independent sets are of size less than $sn$). 
Note that $0 \leq s < c \leq 1$ 
for otherwise the problem is same as the \lang{INDSET} problem. 

Let us define the notion of an approximation algorithm for \lang{MAXINDSET}. 
An $\epsilon$-approximation $A$ for \lang{MAXINDSET} is an algorithm that 
takes a graph $G$ as input and yields an independent set of size at least 
$\epsilon k$ where $k$ is the size  of the maximum independent set in $G$. 

Now we connect the existence of good approximation algorithms for 
\lang{MAXINDSET} to algorithms solving \lang{GAPINDSET}. Note that if 
$A$ is an $\epsilon$-approximation for \lang{MAXINDSET}, then $A$ can be 
used to solve \lang{GAPINDSET(s,c)} where $s < \epsilon c$. For example, let 
us take $\epsilon = 1/2$. Suppose $A$ is a $1/2$-approximation for 
\lang{MAXINDSET}. Then we can use $A$ to solve \lang{GAPINDSET(c/2, c)}. We 
run $A$ on the input graph $G$ and output ``yes'' iff $A$ outputs an independent 
set of size greater than $(c/2)n$. If $G$ had an independent set of size at least 
$cn$, then $A$ is guaranteed to output an independent set of size at least $(c/2)n$. 
Otherwise, by the promise, the largest independent set in $G$ has size less than $(c/2)n$ 
and $A$ outputs an independent set of size less than $(c/2)n$. This shows that if a 
$1/2$-approximation to \lang{MAXINDSET} exists, then \lang{GAPINDSET(c/2, c)} can be solved in 
polynomial time. In otherwords, by showing that \lang{GAPINDSET(s, c)}, where $s/c < \epsilon$, 
is \NP-complete, we may conclude that an $\epsilon$-approximation to \lang{MAXINDSET} does not 
exist unless $\P = \NP$.

\section{\lang{qGAPCSP} $\leq_{m}^{p}$ \lang{GAPINDSET(m, m/2)}}
We now present a reduction from \lang{qGAPCSP} to \lang{GAPINDSET(m, m/2)}. Here $q$ stands for the 
number of variables in each constraint of CSP. The parameter $m$ is the number of constraints. 
The promise in \lang{qGAPCSP} problem is that either all constraints can be satisfied or less than 
$1/2$ the fraction of the constraints can be satisfied (This is where the $m/2$ comes from in \lang{GAPINDSET(m,m/2)}).
The following algorithm constructs a graph from an instance of \lang{qGAPCSP}.

\medskip
{\tt \obeylines \obeyspaces
1. Create $m$ clusters of vertices, one for each constraint. 
2. The vertices in cluster $i$ are in one-to-one correspondence with
.. satisfying assignments for $\psi_{i}$. That is, for each (global) assignment 
.. that satisfies $\psi_{i}$, we add a vertex to cluster $i$ that corresponds to the 
.. restriction of the global assignment satisfying $\psi_{i}$.
3. All vertices within a cluster are connected. Two vertices 
.. $u$ and $v$ in different clusters are connected iff they are not 
.. contradictory. That is, there does not exist any $x_i$ such 
.. that $x_i = 1$ in $u$ and $x_i = 0$ in $v$ or viceversa.
}
\medskip

The running time of the algorithm is polynomial since each cluster contains at most 
$2^q$ vertices and there are only a linear number of clusters.

We now prove the correctness of the reduction. Suppose $\psi$ is a yes instance. 
Then we claim there is an independent set of size $m$. Let $x$ be the (global) assignment 
that satisfies $\psi$. Then, $x$ satisfies each $\psi_{i}$. Choose the vertex 
corresponding to $x$ from the $i^\textrm{th}$ cluster for each $i$. Since the assignment from 
each cluster is the same, there is no edge between any of the vertices. Now suppose $\psi$ was a 
no instance. We claim that no independent set of size $m/2$ exists in $G$. Suppose we were able to 
select $m/2$ vertices. Then, by construction each vertex would be from a different cluster. We are 
also guaranteed that they are not contradictory. So there exists a way to extend the partial 
assignment to yield a global assignment satisfying $m/2$ constraints which violates our assumption 
that $\psi$ is a no instance of the promise problem.

The above result combined with hardness of \lang{qGAPCSP} shows the inapproximability of \lang{MAXINDSET}.

\section{Towards the PCP theorem}
As a first step towards proving the PCP theorem $\NP \subseteq \PCP(O(\log n), O(1))$ we prove the result 
$\lang{LIN} \in \PCP(O(\log n), O(1))$ where \lang{LIN} is the language of all linear system of equations 
solvable over $\mathbb{F}_2$. Assume that the proof $\Pi$ is the satisfying assigment. Then it seems impossible 
to verify with high probability the correctness by looking at only a constant number of bits. We get around 
this problem by demanding a different sort of proof from the prover \footnote{The proof in long-code form is simply the 
Hadamard encoding of the satisfying assignment}. Note that the system could be written as 
$Ax = b$ where $A \in \mathbb{F}_{2}^{m \times n}$, $x, b \in \mathbb{F}_{2}^{n \times 1}$. Suppose the 
system is solvable, then for any $r \in \mathbb{F}_{2}^{m \times 1}$, we have $r^{T}Ax = r^{T}b$. If we 
let $r^{T}A = a$, we may rewrite this as $a.x = r^{T}b$. Note that the right hand side could be computed 
without looking at the proof. We now describe the structure of the proof $\Pi$. 
The proof $\Pi \in \mathbb{F}_{2}^{2^m}$ where the $i^{\textrm{th}}$ bit of $\Pi$ is the value 
of $r^{T}Ax$ for the $i^{\textrm{th}}$ $r$. If the system is satisfiable, an honest prover could compute $a.x$ for 
each choice of $a$ with the satisfying assignment $x$. In the next lecture, we will see that if the system is 
unsatisfiable, then verifier has a strategy to reject with high probability.

\end{document}
