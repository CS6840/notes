\documentclass[11pt]{article}
%%%%%%%%%%%%%%%%%%%%%%%%%%%%%%%%%%%%%%%%%%%%%%%%%%%%%%
%
% This file should be called  preamble.tex 
% Your main LaTeX file should look like :
%
%	\documentclass[11pt]{article}
%	\usepackage{amsmath,amssymb,amsfonts,amsthm,complexity}
%
%	%%%%%%%%%%%%%%%%%%%%%%%%%%%%%%%%%%%%%%%%%%%%%%%%%%%%%%
%
% This file should be called  preamble.tex 
% Your main LaTeX file should look like :
%
%	\documentclass[11pt]{article}
%	\usepackage{amsmath,amssymb,amsfonts,amsthm,complexity}
%
%	%%%%%%%%%%%%%%%%%%%%%%%%%%%%%%%%%%%%%%%%%%%%%%%%%%%%%%
%
% This file should be called  preamble.tex 
% Your main LaTeX file should look like :
%
%	\documentclass[11pt]{article}
%	\usepackage{amsmath,amssymb,amsfonts,amsthm,complexity}
%
%	\input{preamble.tex}
%	\begin{document}
%
%	\lecture{LECTURE NO.}{TITLE OF SCRIBE}{DATE}{Jayalal Sarma M.N.}{YOUR NAME}
%	\theme{THEME FOCUS}
%	\lectureplan{A BRIEF DESCRIPTION OF THE LECTURE}
%	
%	\section{TOPIC 1}
%	...
%	\section{TOPIC 2}
% 	...
%	\end{document}
% If there is not title, leave it as {}
 

\newtheorem{theorem}{Theorem}
\newtheorem{corollary}[theorem]{Corollary}
\newtheorem{lemma}[theorem]{Lemma}
\newtheorem{observation}[theorem]{Observation}
\newtheorem{proposition}[theorem]{Proposition}
\newtheorem{claim}[theorem]{Claim}
\newtheorem{fact}[theorem]{Fact}
\newtheorem{example}[theorem]{Example}
\newtheorem{assumption}[theorem]{Assumption}

\theoremstyle{definition}
\newtheorem{definition}[theorem]{Definition}

\theoremstyle{remark}
\newtheorem{remark}[theorem]{Remark}

% Setting theorem style back for theorems defined in main file.
\theoremstyle{plain}

%\newenvironment{proof}{\noindent{\bf Proof}\hspace*{1em}}{\qed\bigskip}
\newenvironment{proof-sketch}{\noindent{\bf Sketch of Proof}\hspace*{1em}}{\qed\bigskip}
\newenvironment{proof-idea}{\noindent{\bf Proof Idea}\hspace*{1em}}{\qed\bigskip}
\newenvironment{proof-of-lemma}[1]{\noindent{\bf Proof of Lemma #1}\hspace*{1em}}{\qed\bigskip}
\newenvironment{proof-attempt}{\noindent{\bf Proof Attempt}\hspace*{1em}}{\qed\bigskip}
\newenvironment{proofof}[1]{\noindent{\bf Proof}
of #1:\hspace*{1em}}{\qed\bigskip}

%%%%%%%%%%%%%%%%%%%%%%%%%%%%%%%%%%%%%%%%%%%%%%%%%%%
% Useful Complexity Classes
%%%%%%%%%%%%%%%%%%%%%%%%%%%%%%%%%%%%%%%%%%%%%%%%%%%
\newcommand{\ntime}{\hbox{NTIME}}
\newcommand{\nspace}{\hbox{NSPACE}}
\newcommand{\conspace}{\hbox{co-NSPACE}}
\newcommand{\np}{\hbox{NP}}
\newcommand{\pspace}{\hbox{PSPACE}}
\newcommand{\lspace}{\hbox{L}}
\newcommand{\conp}{\hbox{coNP}}
\newcommand{\exptime}{\hbox{EXPTIME}}
\newcommand{\elem}{\hbox{E}}
\newcommand{\nl}{\hbox{NL}}
\newcommand{\bpp}{\hbox{BPP}}
\newcommand{\nregexp}{\hbox{NREGEXP}}
\newcommand{\tqbf}{\hbox{TQBF}}
\newcommand{\threesat}{\hbox{3SAT}}
\newcommand{\cvp}{\hbox{CVP}}
\newcommand{\stconn}{\hbox{STCONN}}
\newcommand{\ispath}{\hbox{ISPATH}}

%\newcommand{\class}{\hbox{$\mathbb{C}$}} 
%\newcommand{\class}{\hbox{$\mathbf{C}$}} 

\newcommand{\lep}{\leq _{\hbox{P}}}
\newcommand{\lel}{\leq _{\hbox{L}}}
\newcommand{\aspace}[1]{{\rm ASPACE}(#1)}
\newcommand{\atime}[1]{{\rm ATIME}(#1)}
\newcommand{\dtime}[1]{{\rm DTIME}(#1)}
\newcommand{\spa}[1]{{\rm SPACE}(#1)}
\newcommand{\ti}[1]{{\rm TIME}(#1)}
\newcommand{\ap}{{\rm AP}}
\newcommand{\al}{{\rm AL}}


%%%%%%%%%%%%%%%%%%%%%%%%%%%%%%%%%%%%%%%%%%%%%%%%%%%
% Useful Macros
%%%%%%%%%%%%%%%%%%%%%%%%%%%%%%%%%%%%%%%%%%%%%%%%%%%
\renewcommand{\Pr}[1]{\mathify{\mbox{Pr}\left[#1\right]}}
\newcommand{\Exp}[1]{\mathify{\mbox{Exp}\left[#1\right]}}
\newcommand{\bigO}O
\newcommand{\set}[1]{\mathify{\left\{ #1 \right\}}}
\def\half{\frac{1}{2}}

\def\implies{\Rightarrow}
\def\prob#1#2{{\mathop{{\rm Prob}}_{#1}}\left[#2 \right]}
\def\var#1#2{{\mathop{{\rm Var}}_{#1}}[#2]}
\def\expec#1#2{{\mathop{{\rm E}}_{#1}}[#2]}
\def\sizeof#1{\left| #1\right|}
\def\setof#1{\left\{ #1\right\}  }

\newcommand{\F}{{\mathbb{F}}}
\newcommand{\Z}{{\mathbb{Z}}}
%\newcommand{\qed}{\rule{7pt}{7pt}}

% \makeatletter
% \@addtoreset{figure}{section}
% \@addtoreset{table}{section}
% \@addtoreset{equation}{section}
% \makeatother

\newcommand{\FOR}{{\bf for}}
\newcommand{\TO}{{\bf to}}
\newcommand{\DO}{{\bf do}}
\newcommand{\WHILE}{{\bf while}}
\newcommand{\AND}{{\bf and}}
\newcommand{\IF}{{\bf if}}
\newcommand{\THEN}{{\bf then}}
\newcommand{\ELSE}{{\bf else}}
\newcommand{\N}{\mathbb{N}}

% \renewcommand{\thefigure}{\thesection.\arabic{figure}}
% \renewcommand{\thetable}{\thesection.\arabic{table}}
% \renewcommand{\theequation}{\thesection.\arabic{equation}}

% Calligraphic letters
\newcommand{\calA}{{\cal A}}
\newcommand{\calB}{{\cal B}}
\newcommand{\calC}{{\cal C}}
\newcommand{\calD}{{\cal D}}
\newcommand{\calE}{{\cal E}}
\newcommand{\calF}{{\cal F}}
\newcommand{\calG}{{\cal G}}
\newcommand{\calH}{{\cal H}}
\newcommand{\calI}{{\cal I}}
\newcommand{\calJ}{{\cal J}}
\newcommand{\calK}{{\cal K}}
\newcommand{\calL}{{\cal L}}
\newcommand{\calM}{{\cal M}}
\newcommand{\calN}{{\cal N}}
\newcommand{\calO}{{\cal O}}
\newcommand{\calP}{{\cal P}}
\newcommand{\calQ}{{\cal Q}}
\newcommand{\calR}{{\cal R}}
\newcommand{\calS}{{\cal S}}
\newcommand{\calT}{{\cal T}}
\newcommand{\calU}{{\cal U}}
\newcommand{\calV}{{\cal V}}
\newcommand{\calW}{{\cal W}}
\newcommand{\calX}{{\cal X}}
\newcommand{\calY}{{\cal Y}}
\newcommand{\calZ}{{\cal Z}}


% Some macro's from Speilman's course.

\makeatletter
\def\fnum@figure{{\bf Figure \thefigure}}
\def\fnum@table{{\bf Table \thetable}}
\long\def\@mycaption#1[#2]#3{\addcontentsline{\csname
  ext@#1\endcsname}{#1}{\protect\numberline{\csname 
  the#1\endcsname}{\ignorespaces #2}}\par
  \begingroup
    \@parboxrestore
    \small
    \@makecaption{\csname fnum@#1\endcsname}{\ignorespaces #3}\par
  \endgroup}
\def\mycaption{\refstepcounter\@captype \@dblarg{\@mycaption\@captype}}
\makeatother

\newcommand{\figcaption}[1]{\mycaption[]{#1}}
\newcommand{\tabcaption}[1]{\mycaption[]{#1}}


%%%%%%%%%%%%%%%%%%%%%%%%%%%%%%%%%%%%%%%%%%%%%%%%%%%%%%%%%%%%%%%%%%%%%%
% Feel free to ignore the rest of this file.

%%%%%%%%%%%%%%%%%%%%%%%%%%%%%%%
% Margins and page indentations
% DO NOT EDIT
%%%%%%%%%%%%%%%%%%%%%%%%%%%%%%
\textwidth=6in
\oddsidemargin=0.25in
\evensidemargin=0.25in
\topmargin=-0.1in
\footskip=0.8in
\parindent=0.0cm
\parskip=0.3cm
\textheight=8.00in
\setcounter{tocdepth} {3}
\setcounter{secnumdepth} {2}
\sloppy


\newcommand{\handout}[5]{
   \renewcommand{\thepage}{#1-\arabic{page}}
   \noindent
   \begin{center}
   \framebox{
      \vbox{
    \hbox to 5.78in { {\bf CS6840: Advanced Complexity Theory} \hfill #2 }
       \vspace{4mm}
       \hbox to 5.78in { {\Large \hfill #5  \hfill} }
       \vspace{2mm}
       \hbox to 5.78in { {\em #3 \hfill #4} }
      }
   }
   \end{center}
   \vspace*{4mm}
}

\newcommand{\lecture}[5]{\handout{#1}{#3}{Lecturer:~#4}{Scribe: #5}{Lecture~#1~: #2}}

\newcommand{\lectureplan}[1]{{\sc Lecture Plan:}#1\\}
\newcommand{\theme}[1]{{\sc Theme:} #1\\}

%	\begin{document}
%
%	\lecture{LECTURE NO.}{TITLE OF SCRIBE}{DATE}{Jayalal Sarma M.N.}{YOUR NAME}
%	\theme{THEME FOCUS}
%	\lectureplan{A BRIEF DESCRIPTION OF THE LECTURE}
%	
%	\section{TOPIC 1}
%	...
%	\section{TOPIC 2}
% 	...
%	\end{document}
% If there is not title, leave it as {}
 

\newtheorem{theorem}{Theorem}
\newtheorem{corollary}[theorem]{Corollary}
\newtheorem{lemma}[theorem]{Lemma}
\newtheorem{observation}[theorem]{Observation}
\newtheorem{proposition}[theorem]{Proposition}
\newtheorem{claim}[theorem]{Claim}
\newtheorem{fact}[theorem]{Fact}
\newtheorem{example}[theorem]{Example}
\newtheorem{assumption}[theorem]{Assumption}

\theoremstyle{definition}
\newtheorem{definition}[theorem]{Definition}

\theoremstyle{remark}
\newtheorem{remark}[theorem]{Remark}

% Setting theorem style back for theorems defined in main file.
\theoremstyle{plain}

%\newenvironment{proof}{\noindent{\bf Proof}\hspace*{1em}}{\qed\bigskip}
\newenvironment{proof-sketch}{\noindent{\bf Sketch of Proof}\hspace*{1em}}{\qed\bigskip}
\newenvironment{proof-idea}{\noindent{\bf Proof Idea}\hspace*{1em}}{\qed\bigskip}
\newenvironment{proof-of-lemma}[1]{\noindent{\bf Proof of Lemma #1}\hspace*{1em}}{\qed\bigskip}
\newenvironment{proof-attempt}{\noindent{\bf Proof Attempt}\hspace*{1em}}{\qed\bigskip}
\newenvironment{proofof}[1]{\noindent{\bf Proof}
of #1:\hspace*{1em}}{\qed\bigskip}

%%%%%%%%%%%%%%%%%%%%%%%%%%%%%%%%%%%%%%%%%%%%%%%%%%%
% Useful Complexity Classes
%%%%%%%%%%%%%%%%%%%%%%%%%%%%%%%%%%%%%%%%%%%%%%%%%%%
\newcommand{\ntime}{\hbox{NTIME}}
\newcommand{\nspace}{\hbox{NSPACE}}
\newcommand{\conspace}{\hbox{co-NSPACE}}
\newcommand{\np}{\hbox{NP}}
\newcommand{\pspace}{\hbox{PSPACE}}
\newcommand{\lspace}{\hbox{L}}
\newcommand{\conp}{\hbox{coNP}}
\newcommand{\exptime}{\hbox{EXPTIME}}
\newcommand{\elem}{\hbox{E}}
\newcommand{\nl}{\hbox{NL}}
\newcommand{\bpp}{\hbox{BPP}}
\newcommand{\nregexp}{\hbox{NREGEXP}}
\newcommand{\tqbf}{\hbox{TQBF}}
\newcommand{\threesat}{\hbox{3SAT}}
\newcommand{\cvp}{\hbox{CVP}}
\newcommand{\stconn}{\hbox{STCONN}}
\newcommand{\ispath}{\hbox{ISPATH}}

%\newcommand{\class}{\hbox{$\mathbb{C}$}} 
%\newcommand{\class}{\hbox{$\mathbf{C}$}} 

\newcommand{\lep}{\leq _{\hbox{P}}}
\newcommand{\lel}{\leq _{\hbox{L}}}
\newcommand{\aspace}[1]{{\rm ASPACE}(#1)}
\newcommand{\atime}[1]{{\rm ATIME}(#1)}
\newcommand{\dtime}[1]{{\rm DTIME}(#1)}
\newcommand{\spa}[1]{{\rm SPACE}(#1)}
\newcommand{\ti}[1]{{\rm TIME}(#1)}
\newcommand{\ap}{{\rm AP}}
\newcommand{\al}{{\rm AL}}


%%%%%%%%%%%%%%%%%%%%%%%%%%%%%%%%%%%%%%%%%%%%%%%%%%%
% Useful Macros
%%%%%%%%%%%%%%%%%%%%%%%%%%%%%%%%%%%%%%%%%%%%%%%%%%%
\renewcommand{\Pr}[1]{\mathify{\mbox{Pr}\left[#1\right]}}
\newcommand{\Exp}[1]{\mathify{\mbox{Exp}\left[#1\right]}}
\newcommand{\bigO}O
\newcommand{\set}[1]{\mathify{\left\{ #1 \right\}}}
\def\half{\frac{1}{2}}

\def\implies{\Rightarrow}
\def\prob#1#2{{\mathop{{\rm Prob}}_{#1}}\left[#2 \right]}
\def\var#1#2{{\mathop{{\rm Var}}_{#1}}[#2]}
\def\expec#1#2{{\mathop{{\rm E}}_{#1}}[#2]}
\def\sizeof#1{\left| #1\right|}
\def\setof#1{\left\{ #1\right\}  }

\newcommand{\F}{{\mathbb{F}}}
\newcommand{\Z}{{\mathbb{Z}}}
%\newcommand{\qed}{\rule{7pt}{7pt}}

% \makeatletter
% \@addtoreset{figure}{section}
% \@addtoreset{table}{section}
% \@addtoreset{equation}{section}
% \makeatother

\newcommand{\FOR}{{\bf for}}
\newcommand{\TO}{{\bf to}}
\newcommand{\DO}{{\bf do}}
\newcommand{\WHILE}{{\bf while}}
\newcommand{\AND}{{\bf and}}
\newcommand{\IF}{{\bf if}}
\newcommand{\THEN}{{\bf then}}
\newcommand{\ELSE}{{\bf else}}
\newcommand{\N}{\mathbb{N}}

% \renewcommand{\thefigure}{\thesection.\arabic{figure}}
% \renewcommand{\thetable}{\thesection.\arabic{table}}
% \renewcommand{\theequation}{\thesection.\arabic{equation}}

% Calligraphic letters
\newcommand{\calA}{{\cal A}}
\newcommand{\calB}{{\cal B}}
\newcommand{\calC}{{\cal C}}
\newcommand{\calD}{{\cal D}}
\newcommand{\calE}{{\cal E}}
\newcommand{\calF}{{\cal F}}
\newcommand{\calG}{{\cal G}}
\newcommand{\calH}{{\cal H}}
\newcommand{\calI}{{\cal I}}
\newcommand{\calJ}{{\cal J}}
\newcommand{\calK}{{\cal K}}
\newcommand{\calL}{{\cal L}}
\newcommand{\calM}{{\cal M}}
\newcommand{\calN}{{\cal N}}
\newcommand{\calO}{{\cal O}}
\newcommand{\calP}{{\cal P}}
\newcommand{\calQ}{{\cal Q}}
\newcommand{\calR}{{\cal R}}
\newcommand{\calS}{{\cal S}}
\newcommand{\calT}{{\cal T}}
\newcommand{\calU}{{\cal U}}
\newcommand{\calV}{{\cal V}}
\newcommand{\calW}{{\cal W}}
\newcommand{\calX}{{\cal X}}
\newcommand{\calY}{{\cal Y}}
\newcommand{\calZ}{{\cal Z}}


% Some macro's from Speilman's course.

\makeatletter
\def\fnum@figure{{\bf Figure \thefigure}}
\def\fnum@table{{\bf Table \thetable}}
\long\def\@mycaption#1[#2]#3{\addcontentsline{\csname
  ext@#1\endcsname}{#1}{\protect\numberline{\csname 
  the#1\endcsname}{\ignorespaces #2}}\par
  \begingroup
    \@parboxrestore
    \small
    \@makecaption{\csname fnum@#1\endcsname}{\ignorespaces #3}\par
  \endgroup}
\def\mycaption{\refstepcounter\@captype \@dblarg{\@mycaption\@captype}}
\makeatother

\newcommand{\figcaption}[1]{\mycaption[]{#1}}
\newcommand{\tabcaption}[1]{\mycaption[]{#1}}


%%%%%%%%%%%%%%%%%%%%%%%%%%%%%%%%%%%%%%%%%%%%%%%%%%%%%%%%%%%%%%%%%%%%%%
% Feel free to ignore the rest of this file.

%%%%%%%%%%%%%%%%%%%%%%%%%%%%%%%
% Margins and page indentations
% DO NOT EDIT
%%%%%%%%%%%%%%%%%%%%%%%%%%%%%%
\textwidth=6in
\oddsidemargin=0.25in
\evensidemargin=0.25in
\topmargin=-0.1in
\footskip=0.8in
\parindent=0.0cm
\parskip=0.3cm
\textheight=8.00in
\setcounter{tocdepth} {3}
\setcounter{secnumdepth} {2}
\sloppy


\newcommand{\handout}[5]{
   \renewcommand{\thepage}{#1-\arabic{page}}
   \noindent
   \begin{center}
   \framebox{
      \vbox{
    \hbox to 5.78in { {\bf CS6840: Advanced Complexity Theory} \hfill #2 }
       \vspace{4mm}
       \hbox to 5.78in { {\Large \hfill #5  \hfill} }
       \vspace{2mm}
       \hbox to 5.78in { {\em #3 \hfill #4} }
      }
   }
   \end{center}
   \vspace*{4mm}
}

\newcommand{\lecture}[5]{\handout{#1}{#3}{Lecturer:~#4}{Scribe: #5}{Lecture~#1~: #2}}

\newcommand{\lectureplan}[1]{{\sc Lecture Plan:}#1\\}
\newcommand{\theme}[1]{{\sc Theme:} #1\\}

%	\begin{document}
%
%	\lecture{LECTURE NO.}{TITLE OF SCRIBE}{DATE}{Jayalal Sarma M.N.}{YOUR NAME}
%	\theme{THEME FOCUS}
%	\lectureplan{A BRIEF DESCRIPTION OF THE LECTURE}
%	
%	\section{TOPIC 1}
%	...
%	\section{TOPIC 2}
% 	...
%	\end{document}
% If there is not title, leave it as {}
 

\newtheorem{theorem}{Theorem}
\newtheorem{corollary}[theorem]{Corollary}
\newtheorem{lemma}[theorem]{Lemma}
\newtheorem{observation}[theorem]{Observation}
\newtheorem{proposition}[theorem]{Proposition}
\newtheorem{claim}[theorem]{Claim}
\newtheorem{fact}[theorem]{Fact}
\newtheorem{example}[theorem]{Example}
\newtheorem{assumption}[theorem]{Assumption}

\theoremstyle{definition}
\newtheorem{definition}[theorem]{Definition}

\theoremstyle{remark}
\newtheorem{remark}[theorem]{Remark}

% Setting theorem style back for theorems defined in main file.
\theoremstyle{plain}

%\newenvironment{proof}{\noindent{\bf Proof}\hspace*{1em}}{\qed\bigskip}
\newenvironment{proof-sketch}{\noindent{\bf Sketch of Proof}\hspace*{1em}}{\qed\bigskip}
\newenvironment{proof-idea}{\noindent{\bf Proof Idea}\hspace*{1em}}{\qed\bigskip}
\newenvironment{proof-of-lemma}[1]{\noindent{\bf Proof of Lemma #1}\hspace*{1em}}{\qed\bigskip}
\newenvironment{proof-attempt}{\noindent{\bf Proof Attempt}\hspace*{1em}}{\qed\bigskip}
\newenvironment{proofof}[1]{\noindent{\bf Proof}
of #1:\hspace*{1em}}{\qed\bigskip}

%%%%%%%%%%%%%%%%%%%%%%%%%%%%%%%%%%%%%%%%%%%%%%%%%%%
% Useful Complexity Classes
%%%%%%%%%%%%%%%%%%%%%%%%%%%%%%%%%%%%%%%%%%%%%%%%%%%
\newcommand{\ntime}{\hbox{NTIME}}
\newcommand{\nspace}{\hbox{NSPACE}}
\newcommand{\conspace}{\hbox{co-NSPACE}}
\newcommand{\np}{\hbox{NP}}
\newcommand{\pspace}{\hbox{PSPACE}}
\newcommand{\lspace}{\hbox{L}}
\newcommand{\conp}{\hbox{coNP}}
\newcommand{\exptime}{\hbox{EXPTIME}}
\newcommand{\elem}{\hbox{E}}
\newcommand{\nl}{\hbox{NL}}
\newcommand{\bpp}{\hbox{BPP}}
\newcommand{\nregexp}{\hbox{NREGEXP}}
\newcommand{\tqbf}{\hbox{TQBF}}
\newcommand{\threesat}{\hbox{3SAT}}
\newcommand{\cvp}{\hbox{CVP}}
\newcommand{\stconn}{\hbox{STCONN}}
\newcommand{\ispath}{\hbox{ISPATH}}

%\newcommand{\class}{\hbox{$\mathbb{C}$}} 
%\newcommand{\class}{\hbox{$\mathbf{C}$}} 

\newcommand{\lep}{\leq _{\hbox{P}}}
\newcommand{\lel}{\leq _{\hbox{L}}}
\newcommand{\aspace}[1]{{\rm ASPACE}(#1)}
\newcommand{\atime}[1]{{\rm ATIME}(#1)}
\newcommand{\dtime}[1]{{\rm DTIME}(#1)}
\newcommand{\spa}[1]{{\rm SPACE}(#1)}
\newcommand{\ti}[1]{{\rm TIME}(#1)}
\newcommand{\ap}{{\rm AP}}
\newcommand{\al}{{\rm AL}}


%%%%%%%%%%%%%%%%%%%%%%%%%%%%%%%%%%%%%%%%%%%%%%%%%%%
% Useful Macros
%%%%%%%%%%%%%%%%%%%%%%%%%%%%%%%%%%%%%%%%%%%%%%%%%%%
\renewcommand{\Pr}[1]{\mathify{\mbox{Pr}\left[#1\right]}}
\newcommand{\Exp}[1]{\mathify{\mbox{Exp}\left[#1\right]}}
\newcommand{\bigO}O
\newcommand{\set}[1]{\mathify{\left\{ #1 \right\}}}
\def\half{\frac{1}{2}}

\def\implies{\Rightarrow}
\def\prob#1#2{{\mathop{{\rm Prob}}_{#1}}\left[#2 \right]}
\def\var#1#2{{\mathop{{\rm Var}}_{#1}}[#2]}
\def\expec#1#2{{\mathop{{\rm E}}_{#1}}[#2]}
\def\sizeof#1{\left| #1\right|}
\def\setof#1{\left\{ #1\right\}  }

\newcommand{\F}{{\mathbb{F}}}
\newcommand{\Z}{{\mathbb{Z}}}
%\newcommand{\qed}{\rule{7pt}{7pt}}

% \makeatletter
% \@addtoreset{figure}{section}
% \@addtoreset{table}{section}
% \@addtoreset{equation}{section}
% \makeatother

\newcommand{\FOR}{{\bf for}}
\newcommand{\TO}{{\bf to}}
\newcommand{\DO}{{\bf do}}
\newcommand{\WHILE}{{\bf while}}
\newcommand{\AND}{{\bf and}}
\newcommand{\IF}{{\bf if}}
\newcommand{\THEN}{{\bf then}}
\newcommand{\ELSE}{{\bf else}}
\newcommand{\N}{\mathbb{N}}

% \renewcommand{\thefigure}{\thesection.\arabic{figure}}
% \renewcommand{\thetable}{\thesection.\arabic{table}}
% \renewcommand{\theequation}{\thesection.\arabic{equation}}

% Calligraphic letters
\newcommand{\calA}{{\cal A}}
\newcommand{\calB}{{\cal B}}
\newcommand{\calC}{{\cal C}}
\newcommand{\calD}{{\cal D}}
\newcommand{\calE}{{\cal E}}
\newcommand{\calF}{{\cal F}}
\newcommand{\calG}{{\cal G}}
\newcommand{\calH}{{\cal H}}
\newcommand{\calI}{{\cal I}}
\newcommand{\calJ}{{\cal J}}
\newcommand{\calK}{{\cal K}}
\newcommand{\calL}{{\cal L}}
\newcommand{\calM}{{\cal M}}
\newcommand{\calN}{{\cal N}}
\newcommand{\calO}{{\cal O}}
\newcommand{\calP}{{\cal P}}
\newcommand{\calQ}{{\cal Q}}
\newcommand{\calR}{{\cal R}}
\newcommand{\calS}{{\cal S}}
\newcommand{\calT}{{\cal T}}
\newcommand{\calU}{{\cal U}}
\newcommand{\calV}{{\cal V}}
\newcommand{\calW}{{\cal W}}
\newcommand{\calX}{{\cal X}}
\newcommand{\calY}{{\cal Y}}
\newcommand{\calZ}{{\cal Z}}


% Some macro's from Speilman's course.

\makeatletter
\def\fnum@figure{{\bf Figure \thefigure}}
\def\fnum@table{{\bf Table \thetable}}
\long\def\@mycaption#1[#2]#3{\addcontentsline{\csname
  ext@#1\endcsname}{#1}{\protect\numberline{\csname 
  the#1\endcsname}{\ignorespaces #2}}\par
  \begingroup
    \@parboxrestore
    \small
    \@makecaption{\csname fnum@#1\endcsname}{\ignorespaces #3}\par
  \endgroup}
\def\mycaption{\refstepcounter\@captype \@dblarg{\@mycaption\@captype}}
\makeatother

\newcommand{\figcaption}[1]{\mycaption[]{#1}}
\newcommand{\tabcaption}[1]{\mycaption[]{#1}}


%%%%%%%%%%%%%%%%%%%%%%%%%%%%%%%%%%%%%%%%%%%%%%%%%%%%%%%%%%%%%%%%%%%%%%
% Feel free to ignore the rest of this file.

%%%%%%%%%%%%%%%%%%%%%%%%%%%%%%%
% Margins and page indentations
% DO NOT EDIT
%%%%%%%%%%%%%%%%%%%%%%%%%%%%%%
\textwidth=6in
\oddsidemargin=0.25in
\evensidemargin=0.25in
\topmargin=-0.1in
\footskip=0.8in
\parindent=0.0cm
\parskip=0.3cm
\textheight=8.00in
\setcounter{tocdepth} {3}
\setcounter{secnumdepth} {2}
\sloppy


\newcommand{\handout}[5]{
   \renewcommand{\thepage}{#1-\arabic{page}}
   \noindent
   \begin{center}
   \framebox{
      \vbox{
    \hbox to 5.78in { {\bf CS6840: Advanced Complexity Theory} \hfill #2 }
       \vspace{4mm}
       \hbox to 5.78in { {\Large \hfill #5  \hfill} }
       \vspace{2mm}
       \hbox to 5.78in { {\em #3 \hfill #4} }
      }
   }
   \end{center}
   \vspace*{4mm}
}

\newcommand{\lecture}[5]{\handout{#1}{#3}{Lecturer:~#4}{Scribe: #5}{Lecture~#1~: #2}}

\newcommand{\lectureplan}[1]{{\sc Lecture Plan:}#1\\}
\newcommand{\theme}[1]{{\sc Theme:} #1\\}


\begin{document}
\lecture{No. 27}{Inapproximability}{Feb 21, 2012}{Jayalal Sarma M N}{Balagopal}
\theme{Inapproximability}
\lectureplan{Inapproximability of Independent set Problem. GAPCSP to GAPIS, 
PCP for LIN, Attempts, Proof in the long-code form. Need of linearity testing.}


Our aim is to show the inapproximability of \lang{MAXINDSET}. For this 
purpose, we introduce the problem \lang{GAPINDSET(s,c)}. An instance 
of \lang{GAPINDSET(s,c)} is a graph $G$ which is guaranteed to either 
have an independent set of size at least $cn$ or to have no independent 
set of size $sn$ (i.e., all independent sets are of size less than $sn$). 
Note that $0 \leq s < c \leq 1$ 
for otherwise the problem is same as the \lang{INDSET} problem. 

Let us define the notion of an approximation algorithm for \lang{MAXINDSET}. 
An $\epsilon$-approximation $A$ for \lang{MAXINDSET} is an algorithm that 
takes a graph $G$ as input and yields an independent set of size at least 
$\epsilon k$ where $k$ is the size  of the maximum independent set in $G$. 

Now we connect the existence of good approximation algorithms for 
\lang{MAXINDSET} to algorithms solving \lang{GAPINDSET}. Note that if 
$A$ is an $\epsilon$-approximation for \lang{MAXINDSET}, then $A$ can be 
used to solve \lang{GAPINDSET(s,c)} where $s < \epsilon c$. For example, let 
us take $\epsilon = 1/2$. Suppose $A$ is a $1/2$-approximation for 
\lang{MAXINDSET}. Then we can use $A$ to solve \lang{GAPINDSET(c/2, c)}. We 
run $A$ on the input graph $G$ and output ``yes'' iff $A$ outputs an independent 
set of size greater than $(c/2)n$. If $G$ had an independent set of size at least 
$cn$, then $A$ is guaranteed to output an independent set of size at least $(c/2)n$. 
Otherwise, by the promise, the largest independent set in $G$ has size less than $(c/2)n$ 
and $A$ outputs an independent set of size less than $(c/2)n$. This shows that if a 
$1/2$-approximation to \lang{MAXINDSET} exists, then \lang{GAPINDSET(c/2, c)} can be solved in 
polynomial time. In otherwords, by showing that \lang{GAPINDSET(s, c)}, where $s/c < \epsilon$, 
is \NP-complete, we may conclude that an $\epsilon$-approximation to \lang{MAXINDSET} does not 
exist unless $\P = \NP$.

\section{\lang{qGAPCSP} $\leq_{m}^{p}$ \lang{GAPINDSET(m, m/2)}}
We now present a reduction from \lang{qGAPCSP} to \lang{GAPINDSET(m, m/2)}. Here $q$ stands for the 
number of variables in each constraint of CSP. The parameter $m$ is the number of constraints. 
The promise in \lang{qGAPCSP} problem is that either all constraints can be satisfied or less than 
$1/2$ the fraction of the constraints can be satisfied (This is where the $m/2$ comes from in \lang{GAPINDSET(m,m/2)}).
The following algorithm constructs a graph from an instance of \lang{qGAPCSP}.

\medskip
{\tt \obeylines \obeyspaces
1. Create $m$ clusters of vertices, one for each constraint. 
2. The vertices in cluster $i$ are in one-to-one correspondence with
.. satisfying assignments for $\psi_{i}$. That is, for each (global) assignment 
.. that satisfies $\psi_{i}$, we add a vertex to cluster $i$ that corresponds to the 
.. restriction of the global assignment satisfying $\psi_{i}$.
3. All vertices within a cluster are connected. Two vertices 
.. $u$ and $v$ in different clusters are connected iff they are not 
.. contradictory. That is, there does not exist any $x_i$ such 
.. that $x_i = 1$ in $u$ and $x_i = 0$ in $v$ or viceversa.
}
\medskip

The running time of the algorithm is polynomial since each cluster contains at most 
$2^q$ vertices and there are only a linear number of clusters.

We now prove the correctness of the reduction. Suppose $\psi$ is a yes instance. 
Then we claim there is an independent set of size $m$. Let $x$ be the (global) assignment 
that satisfies $\psi$. Then, $x$ satisfies each $\psi_{i}$. Choose the vertex 
corresponding to $x$ from the $i^\textrm{th}$ cluster for each $i$. Since the assignment from 
each cluster is the same, there is no edge between any of the vertices. Now suppose $\psi$ was a 
no instance. We claim that no independent set of size $m/2$ exists in $G$. Suppose we were able to 
select $m/2$ vertices. Then, by construction each vertex would be from a different cluster. We are 
also guaranteed that they are not contradictory. So there exists a way to extend the partial 
assignment to yield a global assignment satisfying $m/2$ constraints which violates our assumption 
that $\psi$ is a no instance of the promise problem.

The above result combined with hardness of \lang{qGAPCSP} shows the inapproximability of \lang{MAXINDSET}.

\section{Towards the PCP theorem}
As a first step towards proving the PCP theorem $\NP \subseteq \PCP(O(\log n), O(1))$ we prove the result 
$\lang{LIN} \in \PCP(O(\log n), O(1))$ where \lang{LIN} is the language of all linear system of equations 
solvable over $\mathbb{F}_2$. Assume that the proof $\Pi$ is the satisfying assigment. Then it seems impossible 
to verify with high probability the correctness by looking at only a constant number of bits. We get around 
this problem by demanding a different sort of proof from the prover \footnote{The proof in long-code form is simply the 
Hadamard encoding of the satisfying assignment}. Note that the system could be written as 
$Ax = b$ where $A \in \mathbb{F}_{2}^{m \times n}$, $x, b \in \mathbb{F}_{2}^{n \times 1}$. Suppose the 
system is solvable, then for any $r \in \mathbb{F}_{2}^{m \times 1}$, we have $r^{T}Ax = r^{T}b$. If we 
let $r^{T}A = a$, we may rewrite this as $a.x = r^{T}b$. Note that the right hand side could be computed 
without looking at the proof. We now describe the structure of the proof $\Pi$. 
The proof $\Pi \in \mathbb{F}_{2}^{2^m}$ where the $i^{\textrm{th}}$ bit of $\Pi$ is the value 
of $r^{T}Ax$ for the $i^{\textrm{th}}$ $r$. If the system is satisfiable, an honest prover could compute $a.x$ for 
each choice of $a$ with the satisfying assignment $x$. In the next lecture, we will see that if the system is 
unsatisfiable, then verifier has a strategy to reject with high probability.

\end{document}
