\Lecture{Vamsi Krishna D}{March 02 2012}{31}{Expander Graphs, Degree reduction}
%\theme{Probabilistic Proof Systems}
%\lectureplan{ 
In the last class we have given the outline of Irit Dinur's proof of PCP Theorem in which we viewed SAT as qCSP Problem .We then transformed the q-query CSP to 2-query CSP $\Psi$  .In general, after applying query reduction, the constraint graph constructed from 2-query CSP may have very large degree.Now the goal is to make the degree of the graph constant independent of the instance size , for which we will be using a special type of graphs called expander graphs.
% }
\newline \textbf {Expander Graphs}
A family $G$ = $\{G_{1}, G_{2}, \ldots \}$ of d-regular undirected graphs is an "edge expander family" if there is a constant $\alpha > 0$ such that $\forall G^{'} \in G $, $\forall S\subseteq V(G^{'}) ,|S| \leqslant \frac{1}{2}.|V| $
\newline $E(S,S^c)\geqslant \alpha .d.|S|$
\newline Algebrically expander graphs have intereresting properties.Let $A^{'} = \frac{1}{d}.A$ where A is the adjacency matrix of the expander graph.Because $A^{'}$ is symmetric, the spectral theorem implies that A has n real-valued eigenvalues $\lambda_1 \ge \lambda_2 \ge \cdots \ge \lambda_{n} $ between -1 and 1. Because G is regular, $\lambda_1=1$. In some contexts, the spectral gap of G is defined to be $1-\lambda_2$. In other contexts, the spectral gap is $1-\lambda$, where

    $\lambda=\max\{|\lambda_2|, |\lambda_{n}|\}.$

\textbf {Theorem}
Constant degree expander graphs exist and can be constrcuted explicitly.
\newline We will use this theorem in our analysis without proof.

\textbf{Random walks on Graphs}
A random walk of length k on a  graph G with a root 0 is a stochastic process with random variables $X_1,X_2,\dots,X_k $ such that $X_1=0$ and ${X_{i+1}}$  is a vertex chosen uniformly at random from the neighbors of $X_i$.Let $P_{i}$ be a random vector of dimension n ,where n is the number of vertices in the graph , whose $j^{th}$ component  gives the probabilities of the walk being at that $j^{th}$ vertex after i steps of Random walk.We can see that 

$P_{i+1}=A^{'}P_{i}$

 
The larger the spectral gap ,the smaller the number of walks required to be equally probable of ending at any vertex in graph.

\textbf{Degree Reduction}
Let $G$ be a constraint graph of $\Psi$.We create a new CSP $\Psi^{'}$ by replacing every vertex i in G of
degree $n_i$ by a cloud of $n_i$ vertices. So each variable $x_i$ of $\Psi$ will give rise to $n_i$ variables $x^{'}_{i1} , \ldots , x^{'}_{in_{i}}$ in $\Psi^{'}$ . Each constraint in $\Psi$ gives rise to d/2 parallel constraints in $\Psi^{'}$between unique vertices in
the corresponding clouds. Within each cloud, we interconnect the vertices by a 1/4-edge expander
and make each expander constraint an equality constraint (i.e. requiring that variables get the
same value). Notice that if $\Psi$ has m constraints, then $\Psi^{'}$ will have $m$ variables and $dm$ constraints.
\newline Clearly, if $\Psi$  has a satisfying assignment x, we can obtain a satisfying assignment for $\Psi^{'}$ by setting
$x^{'}_{i1} = \ldots = x^{'}_{in_{i}}= x_i$ for every i.

Now we prove soundness.

\textbf {Theorem}
If some assignment $x^{'}$ the violates at most an $\varepsilon$ -fraction of contraints in $\Psi^{'}$  , then there
exists an assignment $x$ that violates at most a 18$\varepsilon $ fraction of constraints in $\Psi$.

\textbf {Proof}
The assignment $x$ is obtained from $x^{'}$ as follows: Within each cloud, let $x_i$ be the plurality value (i.e.,
the most representative value) among $x^{'}_{i1} , \ldots , x^{'}_{in_{i}}$. Let $\varepsilon_{i}$ be the fraction of constraints violated in cloud i. Then $\Sigma_{i=1}^{n} \varepsilon_i · (dn_{i} /4) ≤ \varepsilon ·(dm/2)$, the total number of violated constraints.

Let’s fix a cloud i. Let $S_i$ be the set of vertices j within this cloud where $x_{ij}^{'}$ agrees with the
plurality assignment. We will argue that, because of the expansion in the cloud, the assignment
within the cloud must largely agree with the plurality assignment. We split the analysis into three
cases:

 i) If $|S_i| > n_i/2$, then by edge expansion $|E(S_i , S_{i}^{c} )| \geq  d|S_{i}^{c}|/4$. Since all the constraints in the cut $(S_i , S_{i}^{c} )$ are violated by the assignment, $|E(S_i , S_{i}^{c} )| \leq  \varepsilon_{i}(dn_i/4)$, so $|S_{i}^{c}| \leq  4ε_i n_i$ .

 ii) If $n_i/4 \leq  |S_i|\leq   n_i/2$, then by edge expansion $|E(S_i , S_{i}^{c} )| \geq  d|S_i|/4 \geq dn_i/16$. Since all the
constraints in the cut are violated, it follows that $\varepsilon_{i} \geq 1/4$, so $|S_{i}^{c}| \leq  n_i \leq 4\varepsilon_i n_i$ .

 iii) If $|S_i | < n_i /4$, then no value in $\Sigma$  is taken more than $\frac{1}{4}$-fraction of the time inside the cloud,
so there must exist some partition of the values within the cloud so that the smaller side of
the partition has between $\frac{n_i}{4}$ and  $\frac{n_i}{2}$ vertices. Just like in the previous case, we get that
$|S_{i}^{c} | \leq  n_i \leq  4\varepsilon_i n_i$ .
In all cases above, $|S_{i}^{c} | \leq  4\varepsilon_i n_i$ for every i.

Now consider what happens in $\Psi^{'}$ when we replace the assignment $x^{'}$ with the plurality assignment
$x_{plur}^{'}$ (i.e. one that equals the plurality of x on every cloud). For each cloud, this may cause the
violation of at most $(d/2)|S_{i}^{'}|$ additional constraints that go out of the cloud. So if $x^{'}$ violates $\varepsilon dm$
constraints in $\Psi^{'}$ , $x_{plur}^{'}$ will violate at most
$ \varepsilon (dm)/2 + \Sigma_{i=1}^{n} (d/2)|S_{i}^{c}|$
$ \leq \varepsilon (dm)/2 + \Sigma_{i=1}^{n} (d/2)4\varepsilon_i n_i$
$ \leq \varepsilon (dm)/2 + 8(\varepsilon (dm)/2 )$
$= 9(\varepsilon (dm)/2 )$ .
constraints of $\Psi^{'}$. This is a 9$\varepsilon $-fraction of all the constraints in $\Psi^{'}$ . So the assignment x can violate at most a 18$\varepsilon $ fraction of constraints in $\Psi $.
\newline So the soundness gap goes down by at most a constant factor in this transformation .
Acknowledgements:
\newline--Lecture notes of Prof.Jayalal Sharma
\newline--Lecture 13 , Theory of Computational Complexity , The Chinese University of Hong Kong Scribe.
\newline--The PCP Theorem by Gap Amplification by Irit Dinur.
